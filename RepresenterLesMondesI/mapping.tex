\documentclass{amsart}
\usepackage[utf8]{inputenc}
    
% \usepackage{wrapfig}

\usepackage{titlesec}
\titleformat{\section}{\normalfont\bf\large}{}{0em}{}
\titleformat{\subsection}{\normalfont\bf}{}{0em}{}

\usepackage{manyfoot}
\DeclareNewFootnote{A}
\DeclareNewFootnote{E}[alph]

\usepackage{xcolor}
\long\def\gray#1{{\color{gray}{#1}}}\let\grey=\gray
\long\def\blue#1{{\color{blue}{#1}}}
\long\def\red#1{{\color{red}{#1}}}

\usepackage{lpic}

\usepackage[hyperfootnotes=false,
    colorlinks,
    linkcolor={blue!80!black},
    citecolor={blue!50!black},
    urlcolor={blue!80!black}]{hyperref}

\def\foothref#1#2{\href{#1}{ #2}\footnoteA{\url{#1}}}

% \newcommand{\skipthis}[2][\dots]{\red{[#1]}}
\newcommand{\skipthis}[2][\dots]{\red{[#1]}\gray{#2}}
\newtheorem*{theorem}{Theorem}

\DeclareMathOperator\distance{\mathit{distance}}

\begin{document}
\centerline{\bf\large Mapping the Worlds I}
\medskip
\centerline{by \'Etienne Ghys}

\begin{center}
  \includegraphics[width=\textwidth]{Pix/TypusOrbisTerrarum}
\end{center}

Cartography has accompanied mathematics since its very beginnings%
\footnoteE{This is evident from the name of the vast area of
  mathematics --- ``Geometry'', that is ``measuring the Earth''} %
and continually renews its set of problems. Originally, it was a
matter of drawing maps as accurately as possible of continents that
were more or less well known. Above is a map dating from 1570, in
which to the south we see a ``Terra Australis Incognita'' --- which
does not exist.
Today, this mission of the cartographer remains relevant, but there
are other new worlds we would like to represent in atlases. Cyberspace
and the brain are only two examples of these modern-day terra
incognita.

In this series of articles, I would like to present a few selected
pieces of this interaction between cartography and mathematics. The
subject is broad, and I will only touch upon it briefly, but I hope to
show through these examples how the two disciplines mutually enrich
one another.

\section{What is a map?}
Here are two maps.

The one on the left is a reconstruction of Anaximander’s map (around
550 BCE). Described by Herodotus, it is one of the oldest maps
representing the world as a whole. The map on the right is the famous
``Carte du Tendre,'' dating from the 17th century, which depicts the
most mysterious land: that of romantic emotions.
\begin{center}
  \includegraphics[width=0.4\textwidth]{Pix/Anaximandre}%
  \includegraphics[width=0.6\textwidth]{Pix/Carte_du_tendre}
\end{center}

\skipthis[Next par is reduced]
A map is an image. It is a representation of a continent, a country, a
region, etc., on a medium --- most often a sheet of paper --- which
one can, for example, slip into one’s pocket when traveling through
the country in question.
If we denote by $X$ the country and by $Y$ the sheet of paper, the map
is a mapping
\[
  f:X\rightarrow Y
\]

\section{What do we expect from a map?}

\skipthis[Section is reduced]{}
That it be faithful, of course! We hope to gather enough information
in $Y$ to find our way around in $X$. We will see later --- though the
reader can already imagine --- that if it is a matter of making a map of
the New York subway, a sheet of paper will do perfectly well, but if we
want to represent the network of connections among the billions of
neurons that make up our brain, we will probably have to look for
another medium!

Faithfulness in our language means injectivity of a mapping $f$.

\section{Distances}

We also expect a map to be accurate. In general, there is so much
information in $X$ that we would like to represent in $Y$, but $Y$ is
only a small sheet of paper and choices have to be made. Here, I will
focus on a single aspect: that the map preserves \blue{relative}
distances, though there would be many other possibilities.

In ordinary cartography, $X$ is a country on Earth, so two points
$x_1$ and $x_2$ of $X$ are separated by a distance that I will denote
$\distance_{X}(x_{1},x_{2})$.  When $Y$ is a sheet of paper, two
of its points $y_{1}$ and $y_{2}$ are likewise separated by a distance
$\distance_{Y}(y_{1},y_{2})$.

An ideal map would be such that the distance between two points at the
source, in $X$, is exactly the same as the one we measure on the map,
in the target. That is
\[
  \distance_{Y}\big(f(x_{1}),f(x_{2})\big)
  =
  \distance_{X}(x_{1},x_{2})
\]
That said, such a map of France would have to be a thousand kilometers
across, which is not very practical. Of course, we use maps at a
certain scale. For example, if we want one centimeter on the map to
represent one kilometer ``in real life,'' that means that distances are
multiplied in the target (on paper) by $k = 0.00001 = 1/100000$. Thus,
for an ideal map at scale $k$ we would instead require:
\[
  \distance_{Y}\big(f(x_{1}),f(x_{2})\big)
  =
  k\cdot\distance_{X}(x_{1},x_{2})
\]



\end{document}