\documentclass{amsart}
\usepackage[utf8]{inputenc}
    
\usepackage{titlesec}
\titleformat{\section}{\normalfont\bf\large}{\relax}{0em}{}
\titleformat{\subsection}{\normalfont\bf}{\relax}{0em}{}

\usepackage{enumitem}
\usepackage{quoting}

\usepackage{xcolor}
\long\def\gray#1{{\color{gray}{#1}}}\let\grey=\gray
\long\def\blue#1{{\color{blue}{#1}}}
\long\def\red#1{{\color{red}{#1}}}

\usepackage{wrapfig}
\usepackage{lpic}

\usepackage{manyfoot}
\DeclareNewFootnote{A}
\DeclareNewFootnote{E}[alph]
\usepackage[hyperfootnotes=false,
    colorlinks,
    linkcolor={blue!80!black},
    citecolor={blue!50!black},
    urlcolor={blue!80!black}]{hyperref}
\def\foothref#1#2{\href{#1}{ #2}\footnoteA{\url{#1}}}

\newcommand{\skipthis}[2][\dots]{\red{[#1]}\gray{#2}}
\newtheorem*{theorem}{Theorem}
\newtheorem*{question}{Question}
\newtheorem*{problem}{Problem}
\newtheorem{exercise}{Exercise}

\let\dmo=\DeclareMathOperator
\dmo\distance{\mathsf{distance}}
\dmo\area{\mathsf{area}}
\dmo\population{\mathsf{population}}
\let\time=\undefined
\dmo\time{\mathsf{time}}
\let\tmp\epsilon
\let\epsilon\varepsilon
\let\varepsilon\tmp


\begin{document}
\centerline{\bf\large Charting the Worlds I%
  \footnoteE{English translation by Rostislav Matveev of the article
    \textit{Repr\'esenter des Mondes}, by
    \'Etienne Ghys, (\red{[YEAR?]}),
    \url{https://images-des-maths.pages.math.cnrs.fr/%
      freeze/Representer-les-mondes.html}.\\
  }}%

\medskip

\centerline{by \'Etienne Ghys}

\begin{center}
  \includegraphics[width=\textwidth]{Pix/TypusOrbisTerrarum}
\end{center}

Cartography has accompanied mathematics since its very beginnings%
\footnoteE{This is also evident from the name of the vast and possibly
  most ancient area of mathematics --- ``geometry'', that is
  ``measuring the Earth.''  (Ed.)} %
and continually renews its set of problems. Originally, it was a
matter of drawing maps as accurately as possible of continents that
were more or less well known. Above is a map dating from 1570, in
which to the south we see a ``Terra Australis Incognita'' --- which
does not exist.  Today, this mission of the cartographer remains
relevant, but there are other new worlds we would like to represent in
atlases. Cyberspace and the brain are only two examples of these
modern-day terra incognita.

In this series of articles, I would like to present a few selected
pieces of this interaction between cartography and mathematics. The
subject is broad, and I will only touch upon it briefly, but I hope to
show through these examples how the two disciplines mutually enrich
one another.

\section{What is a map?}
Here are two maps.  The one on the left is a reconstruction of
Anaximander’s map (around 550 BCE). Described by Herodotus, it is one
of the oldest maps representing the world as a whole. The map on the
right is the famous ``Carte du Tendre,'' dating from the 17th century,
which depicts the most mysterious land: that of romantic emotions.
\begin{center}
  \includegraphics[width=0.4\textwidth]{Pix/Anaximandre}%
  \includegraphics[width=0.6\textwidth]{Pix/Carte_du_tendre}
\end{center}

\skipthis[Next par is reduced]{}
A map is an image. It is a representation of a continent, a country, a
region, and so on, on a medium --- most often a sheet of paper --- which
one can, for example, slip into one’s pocket when traveling through
the country in question.
If we denote by $X$ the country and by $Y$ the sheet of paper, the map
is a mapping
\[
  f:X\rightarrow Y
\]

\section{What do we expect from a map?}

\skipthis[Section is reduced]{}
That it be faithful, of course! We hope to gather enough information
in $Y$ to find our way around in $X$. We will see later --- though the
reader can already imagine --- that if it is a matter of making a map of
the New York subway, a sheet of paper will do perfectly well, but if we
want to represent the network of connections among the billions of
neurons that make up our brain, we will probably have to look for
another medium!

Faithfulness in our language means injectivity of a mapping $f$.

\section{Distances}

We also expect a map to be accurate. In general, there is so much
information in $X$ that we would like to represent in $Y$, but $Y$ is
only a small sheet of paper and choices have to be made. Here, I will
focus on a single aspect: that the map preserves \blue{relative}
distances, though there would be many other possibilities.

In ordinary cartography, $X$ is a country on Earth, so two points
$x_1$ and $x_2$ of $X$ are separated by a distance that I will denote
$\distance_{X}(x_{1},x_{2})$.  When $Y$ is a sheet of paper, two
of its points $y_{1}$ and $y_{2}$ are likewise separated by a distance
$\distance_{Y}(y_{1},y_{2})$.

An ideal map would be such that the distance between two points at the
source, in $X$, is exactly the same as the one we measure on the map,
in the target. That is
\[
  \distance_{Y}\big(f(x_{1}),f(x_{2})\big)
  =
  \distance_{X}(x_{1},x_{2})
\]
That said, such a map of France would have to be a thousand kilometers
across, which is not very practical. Of course, we use maps at a
certain scale. For example, if we want one centimeter on the map to
represent one kilometer ``in real life,'' that means that distances are
multiplied in the target (on paper) by $k = 0.00001 = 1/100000$. Thus,
for an ideal map at scale $k$ we would instead require:
\[
  \distance_{Y}\big(f(x_{1}),f(x_{2})\big)
  =
  k\cdot\distance_{X}(x_{1},x_{2})
\]

\section{The ideal does not exist!}

I claim that there is no map of Europe, for instance, that preserves distances exactly.

To prove this, I consider four cities: Athens, Madrid, Paris, and
Oslo. Here is a table giving the distances between them.%
\footnoteA{I had intended to compute these distances myself from the
  latitudes and longitudes of the cities, but I discovered a website
  that does it for me in just a few clicks\dots The calculations
  assume that the Earth is perfectly spherical, which it is not
  quite\dots, but this does not change anything in the argument.}
\medskip

\begin{center}
  \begin{tabular}{|c||c|c|c|c|}
    \hline
    \mbox{}& Athens & Madrid & Paris & Oslo\\
    \hline\hline
    Athens & 0km & 1743km & 2414km & 2612km \\
    \hline
    Madrid & 1743km & 0km & 936.6km & 2224km \\
    \hline
    Paris  & 2414km & 936.6km & 0km & 1351km \\
    \hline
    Oslo   & 2612km & 2224km & 1351km & 0km \\
    \hline
  \end{tabular}
\end{center}
\medskip

Let us try to construct an exact map of Europe, say at scale
$k = 1/100000$ (one centimeter equals one kilometer).%
\footnoteA{Images are created using the excellent software GeoGebra. %
  \url{https://www.geogebra.org/}} %
To do this, we must first mark the (image of a) first city, say
Athens. This is the point $A$. Next, we place a second city, say
Madrid: the point $M$ must be 17.43 centimetres from $A$. No problem
so far\dots We can place $M$ at any point on a circle, but in any case
all these positions are equivalent, up to rotating the sheet of paper
around $A$.

Next, we place Oslo. We must place a point $O$ at 26.12 cm from $A$
and 22.24 cm from $M$. So we must construct two circles, centered at
$A$ and $M$, with radii 26.12 cm and 22.24 cm, and place $O$ at their
intersection. There are two intersection points, and we may choose
either one, since the other is obtained from the first by
symmetry. Let us therefore choose the one that respects the
orientation we are used to, that is ``to the North''. So far so good.

\begin{center}
  \hfill
  \includegraphics[width=0.4\textwidth]{Pix/dessin1}
  \hfill
  \includegraphics[width=0.4\textwidth]{Pix/dessin}
  \hfill\rule{0mm}{1mm}
\end{center}

It remains to place Paris on the map.
\begin{itemize}[leftmargin=4mm,rightmargin=0mm]
\item We know the distance from Paris to Oslo and to Athens: this
  gives us only two possibilities for placing Paris, at $P_1$ or
  $P_1'$.
\item We know the distance from Paris to Oslo and to Madrid: this also
  gives us only two possibilities for placing Paris, at $P_2$ or
  $P_2'$.
\item We know the distance from Paris to Athens and to Madrid: this
  gives us only two possibilities for placing Paris, at $P_3$ or
  $P_3'$.
\end{itemize}

The problem is that these possibilities are not consistent: the points
$P_1$ and $P_1'$ do not coincide with $P_2$ and $P_2'$\dots They are not
very far from one another, but they do not coincide\dots
\medskip

\centerline{\textbf{It is impossible to draw an exact map of Europe!}}
\medskip

\begin{exercise}
  Find four cities that, unlike the ones that we considered above, can
  be represented in the plane in such a way as to respect the
  distances exactly (up to a scale). Can you find all the situations
  in which this is possible?
\end{exercise}

\begin{exercise}[Difficult]
  Given six positive numbers $d_{1,2}$, $d_{1,3}$, $d_{1,4}$,
  $d_{2,3}$, $d_{2,4}$ and $d_{3,4}$, under what
  condition is it possible to find four points in the plane
  $P_1$, $P_2$, $P_3$, $P_4$ such that the pairwise distances
  $\distance(P_{i}, P_{j})$ between them are exactly $d_{i,j}$?
\end{exercise}

\section{What can we do nevertheless?}
% How can we do as well as possible?
Since there is no perfect map, we have to resign ourselves to
approximations. For example, we can say that a map is precise ``within
10\%'' if the distances measured in $X$ and in $Y$ do not differ by
more than 10\%, taking the scale $k$ into account, of course. Formally
speaking, we will say that a map $f$ has precision $\epsilon$ if:
\begin{align*}
  &\distance\big(f(x_{1}),f(x_{2})\big)
  \leq
    k\cdot(1+\epsilon)\cdot\distance(x_{1},x_{2})
  \intertext{\centering and}
  &\distance(x_{1},x_{2})
  \leq
    \frac1k\cdot(1+\epsilon)\cdot\distance\big(f(x_{1}),f(x_{2})\big)
\end{align*}%
for all points $x_{1}$ and $x_{2}$ of $X$. If $\epsilon = 0$, this
would mean that a map is exact\dots, which does not exist. If
$\epsilon = 0.1$, this corresponds to a map accurate to within $10\%$.

These two formulas express the fact that the map works in both directions:
\begin{itemize}
\item if I know the distance $\distance_X(x_{1},x_2)$ between
  two points ``in reality'', then I also know the
  distance $\distance_Y\big(f(x_1),f(x_2)\big)$ between the points
  that represent $x_{1}$ and $x_{2}$ on the map, up to the precision
  $\epsilon$. I just need to multiply
  by the scale, and the error will be at most the percentage given by $\epsilon$.
\item if I know the distance $\distance_Y\big(f(x_1),f(x_2)\big)$
  between two points on the map, then to find the distance
  $\distance_X(x_1,x_2)$ between the points represented by $f(x_1)$
  and $f(x_{2})$ on the map I have to divide by the scale the result
  will be at most $\epsilon$ percentage off.
\end{itemize}
Of course, we strive for maps for which $\epsilon$ is as small as possible.

The search for accurate maps began as soon as precise measurements of
latitudes, and especially of longitudes, became possible. One anecdote
illustrates the imperfection of maps in the 17th century.
\begin{quoting}[vskip=1ex,leftmargin=5mm,rightmargin=5mm]\sf
  In a map published in the early Mémoires of the Acad\'emie des
  Sciences, vol. VII, p. 430 --- one of the first in which longitudes were
  counted from the Paris Observatory --- the shaded boundaries represent
  those established from astronomical observations carried out, under
  the orders of Louis XIV, by various members of the Acad\'emie des
  Sciences. The boundaries shown in simple outline, with names in
  italics, reproduce the map of the famous Sanson, drawn up in 1679
  and far superior to those that had preceded it. One sees that in
  Sanson’s map the errors were generally considerable, and that
  Brittany was displaced more than 100 kilometers to the west. Thus,
  when Louis XIV saw this map, he complained to the academicians that
  they had considerably reduced the extent of his dominions.%
  \footnoteA{Bigourdan, \textit{La carte de France,} Annales de
    G\'eographie (1899) Volume 8 Num\'ero 42, pp. 427-437, Download:
    \url{https://www.persee.fr/doc/geo_0003-4010_1899_num_8_42_6155}(In
    French)}
\end{quoting}

\section{A theorem and a question of J. Milnor}
John Milnor is an American mathematician who has profoundly influenced
twen\-ti\-eth-century mathematics, particularly in the fields of topology
and dynamics. In 1969, he wrote a nice article%
\footnoteA{J. Milnor, \textit{A problem in cartography.}
  Amer. Math. Monthly 76 (1969) pp. 1101--1112.} %
in which he proves a theorem about cartography and poses a problem\dots
which still, up until today, has no solution.

\begin{theorem}
  Among all maps $f : X \to Y$, there exists at least one with a
  maximal possible precision.
\end{theorem}

This result is an example of what we call a ``compactness theorem''.%
\footnoteA{Indeed, the result follows quite easily from
  Arzel\`a--Ascoli Theorem} %
But it is also a typical example of an existence theorem in
mathematics, which is extremely frustrating for at least two reasons.

The first is that the theorem states that optimal maps exist, but it
does not say how to find them! It is good to know that something
exists, but without any concrete indication of where it can be found,
its appeal is diminished\dots For example, if $X$ is France, how can
one construct such a map?

The second is that this is a theorem of existence and not of
uniqueness. There may be many optimal maps. Of course, starting from a
map $f:X \to Y$, one can translate or rotate the sheet of paper $Y$,
or even change the scale, and thereby produce another map that clearly
has the same precision. But this ``new map'' is really the same one:
we have merely rotated the sheet of paper. Only mathematicians would
consider them different!

\begin{question}
  Up to these operations of rotating the sheet of paper, shifting
  and rescaling, is there a unique optimal map?
\end{question}

If that were the case, how could one construct it, what would it look
like, what would its regularity be? All these are open questions, even
for countries much simpler than France.

For example, consider a ``rectangular'' country $X$, bounded by two
parallels and two meridians. Can one solve the above problems at least
in this particular case? Uniqueness? Practical construction? Nice
challenges for the readers!

\section{A case where everything is understood\dots}
The situation is not completely hopeless. In his article, Milnor
completely solves the question for a circular country. Admittedly,
there are few regions of the world that are circular and that one
would want to map (apart from the polar regions). But at least it is a
simple case in which everything is understood. This seems to be one of
the cases of so called ``Spherical cow in a vacuum''.%
\footnoteA{This alludes to a joke highlighting the oversimplifications
  common in theoretical science: \sf A dairy farm hires a
  mathematician to improve the efficiency of milk production. After
  collecting massive amounts of data at the farm and spending a few
  sleepless nights analyzing it, the mathematician presents his
  report, which begins: {\sl ``Consider the case of a spherical cow in
    a vacuum...''}} %
Nevertheless\dots

\begin{figure}
  \hfill
  \includegraphics[width=0.4\textwidth]{Pix/dessin-2}
  \hfill
  \includegraphics[width=0.48\textwidth]{Pix/dessinc}
  \hfill
  \rule{0pt}{1ex}
\end{figure}
The left figure above is a spherical cap $X$, centered for instance at
the North Pole.
On the right figure the tangent plane $Y$ at the North Pole is shown,
on which we are going to draw the map of $X$. Each point $x$ of $X$
lies on a meridian starting from the North Pole. We then consider the
ray in $Y$ that is tangent to this meridian at the North Pole, and we
place $f(x)$ on this ray, at the same distance from the origin as $x$
is from the North Pole.

\begin{wrapfigure}{r}{0mm}
  \begin{lpic}[t(-4ex),b(-1ex)]{Pix/AzEqHN(0.2)}
  \end{lpic}
\end{wrapfigure}

I stress an important point: in $X$, the distances are
\textit{intrinsic distances} the sphere --- that is, the length of
shortest path drawn on the sphere from one point to another, without
cutting through the interior of the Earth to go faster\dots The
distance in $X$ between $x$ and the North Pole is therefore the length
of the meridian arc from $x$ to the North Pole.
The resulting map is shown on the figure.

In cartography, this map is called the \textit{azimuthal equidistant
  projection}%
\footnoteE{See, for example, the flag of the UN. (Ed.)} %
(attributed to Guillaume Postel), while in mathematical terms, it is
known as the (inverse) \textit{exponential map}.
\begin{theorem} The azimuthal equidistant projection is the only map
  of a spherical cap that has the best possible precision.
\end{theorem}

\section{Milnor Conjecture}

It is not difficult to compute the distortion $\epsilon$ of the azimuthal
equidistant map. The result is a formula:
\[
  \epsilon = \sqrt{\frac{a}{\sin(a)}}-1
\]
where $a$ denotes the aperture of the cap --- the angle shown in the
figure. Here is a small table giving the precision $\epsilon$ for
various values of the angle.
\begin{center}
  \begin{tabular}{|l|c|c|}
    \hline
    Angle $a$ & Distortion $\epsilon$ &  \\
    \hline
     $23^{\circ}26'$ & $1.41\%$ & Arctic polar region \\
    \hline
     $45^{\circ}$ & $5.39\%$ &  \\
    \hline
     $66^{\circ}34'$ & $12.5\%$ & North of Tropic of Cancer \\
    \hline
     $90^{\circ}$ & $25.3\%$ &  Northen Hemisphere\\
    \hline
     $113^{\circ}26'$ & $46.9\%$ & North of Tropic of Capricorn \\
    \hline
     $135^{\circ}$ & $82.5\%$ &  \\
    \hline
     $156^{\circ}34'$ & $162\%$ & North of Antarctic Polar Circle \\
    \hline
     $179^{\circ}$ & $1238\%$ &  \\
    \hline
  \end{tabular}
\end{center}

Instead of computing the precision as a function of the angle $a$, one
can compute it as a function of the number
\[
  u = \frac{\area(\textrm{Cap})}{\area(\textrm{Earth})}
\]
which lies between $0$ and $1$ and represents the proportion of the
Earth covered by the cap.

One obtains a complicated and uninteresting formula, but Milnor shows
that this precision is less (that is, better) than 
\[
  \frac13 u + \frac12 u^2
\]
whenever $u \leq 1/2$, that is, when the cap is smaller than a
hemisphere.%
\footnoteA{In fact, Milnor does not give this estimate, but it follows
  easily from his calculations\dots} %

Milnor proposes a problem: show that the same estimate for the
precision holds for noncircular, convex countries. A country $X$ is
convex if whenever two points $x_1$ and $x_2$ lie in $X$, the shortest
path on the sphere joining them is entirely contained in $X$. In plain
language, this rules out ``fjords''\dots

\begin{problem}
  Let $X$ be a convex country lying in a hemisphere, and let $u$ be
  the proportion of the Earth’s surface occupied by $X$. Is it
  possible to draw a map of $X$ with precision
  $\frac13 u + \frac12 u^2$?
\end{problem}

More than forty years later, this problem still has not been solved.

Here is an example: let us surround the United States of America by a
rectangle. This rectangle covers about $1.5\%$ of the Earth.%
\footnoteA{Such a powerful country that covers such a small percentage
  of the world...} %
According to Milnor, the best known map of the United States is
accurate to within $2.2\%$, whereas according to his conjecture, there
ought to exist one accurate to within $1\%$\dots

The literature on cartography is immense, and many types of maps
exist, each trying to represent one feature or another as well as
possible.%
\footnoteA{For a mathematical approach see, for example, %
  \url{http://www.yann-ollivier.org/carto/carto.php} (In French)}$^,$%
\footnoteE{There are numerous websites devoted to the mathematics of
  cartographic projections. An internet search will produce a list of
  resources to suit any taste. (Ed.)}%

\section{Other trade-offs, other distances\dots}

There are not only distances in kilometers that one might want to
represent on a map. If $x_1$ and $x_2$ are two cities in France, we
can compute the time
\[
  \time(x_1,x_2)
\]
needed to go from $x_1$ to $x_2$. One would need to be more precise,
specifying whether one is traveling by car or by high-speed train, the
time of day and the day of travel, etc. We encounter a new problem:
the space $X$ that we want to represent is not known with complete
precision, and one could even say it is not completely defined. Let us
ignore this difficulty for the moment, promising to return to it in
other articles. A precise temporal map would be very useful: one would
read off in centimeters on the map the distance between $f(x_1)$ and
$f(x_2)$ the time $\time(x_1,x_2)$, expressed for example in
hours.

Of course, for the same reasons we saw above, it is in general
impossible to construct an exact temporal map, and one has to do ``as
well as possible.'' A good exercise, following the reading of Milnor’s
article, is to check that, given a ``temporal'' country $X$, it is
possible to find a map that represents it ``as well as possible,''
with the same issues as before: no idea how to actually construct such
a map!

Here is a temporal map of France for the SNCF.%
\footnoteE{SNCF (\textbf{S}oci\'et\'e \textbf{n}ationale des
  \textbf{c}hemins de \textbf{f}er fran\c{c}ais) is the French railway
  operator. (Ed.)} %

\begin{center}
  \includegraphics[width=0.5\textwidth]{Pix/SNCF}
\end{center}
It was constructed to be exact ``from Paris,'' in the same way that
the azimuthal equidistant map is exact ``from the North Pole.'' It is
therefore an SNCF azimuthal equidistant map.

Along transverse routes, one shouldn’t expect very high precision. How
can one draw a more accurate temporal map of the SNCF network? There’s
work here for mathematicians.

There is another example,%
\footnoteA{Travel Time Tube Map. %
  \url{https://www.tom-carden.co.uk/p5/tube_map_travel_times/applet/}} %
using a cool modern technology. It is an interactive map of the London
Underground network. Each time user chooses a station on the map, the
map deforms to represent ``the azimuthal equidistant projection from
that station.'' In plain language, you can read directly on the map the
travel time from that station.


\subsection{Yet another method}
Suppose we have a large number of points $x_1, x_2, \dots, x_N$ and we
have measured all the distances $d_{i,j}$ between these points. These
distances may be in kilometers or in hours, for that matter. How can
we ``best'' represent this situation by points
$y_1 = f(x_1), y_2 = f(x_2), \dots, y_N = f(x_N)$ in the plane, given
that we have just seen that no method is known for finding the ``best
map,'' the one whose existence is guaranteed by Milnor?

Once we choose the points $y_i$, the defect of the map will be
smaller the closer $\distance_{Y}(y_i,y_j)$ is to $k\cdot d_{i,j}$,
where $k$ is our schoosen scale. One idea is to compute the sum of the
squares of the differences $\distance_{Y}(y_i,y_j) - k\cdot d_{i,j}$:
\[
  \mathsf{defect}:=\sum\big(\distance_{Y}(y_i,y_j) - k\cdot d_{i,j}\big)^{2}
\]
and try to find points $y_i$ for which the defect is as small as
possible.

Why the sum of squares? There are several good reasons for this,
particularly coming from physics, but the best reason is that this new
minimization problem is much simpler than Milnor’s. We know good
methods for finding the $y_i$'s that minimize this sum of squares, and
computers can compute it without difficulty. We have changed our
ambitions; our new problem may be less natural, but at least we know
how to solve it! Cartography is also a matter of ``cookbook recipes.''

\section{Other criteria}

There are many other things one would like to represent on a
map. Distances, even temporal ones, are only one example. Here I would
like to speak about \textit{cartograms}.%
\footnoteE{A cartogram is a map distorted so that the areas of geographic
  regions on the map represents a specific quantity, such as
  population or GDP, instead of their actual land area. Of course,
  such maps are not suitable for navigation, but serve as
  visualization of other features other than distances. (Ed.)} %
Here are some cartograms%
\footnoteA{Images are taken from a very interesting site
  (\url{https://worldmapper.org/}) that offers many other kinds of
  cartograms. } %
representing the world population.
\medskip

\newlength{\mapwidth}\setlength{\mapwidth}{0.5\textwidth-0.5em}
\begin{center}
  \includegraphics[width=\mapwidth]{Pix/1500}
  \includegraphics[width=\mapwidth]{Pix/1900}\\
  \parbox{\mapwidth}{\centerline{\sf World population in 1500}}
  \parbox{\mapwidth}{\centerline{\sf World population in 1900}}\\
  \bigskip\bigskip
  
  \includegraphics[width=\mapwidth]{Pix/1960}
  \includegraphics[width=\mapwidth]{Pix/2050}
  \parbox{\mapwidth}{\centerline{\sf World population in 1960}}
  \parbox{\mapwidth}{\centerline{\sf Projected world population in 2050}}
\end{center}
\medskip

As you have understood, the map $f:X \to Y$ tries to represent the
population of a country $P$ (in millions of inhabitants, for example)
by the area of its image $f(P)$ (in $cm^2$, for example). In formulas,
for every country $P$ in $X$, we would like to have the relation
\[
  \area\big(f(P)\big)=\population(P)
\]
\skipthis[The next paragraphs are heavily edited]{}
\blue{
Here again, the existence of such a map is anything but obvious; it
follows from the theorem, due to Oxtoby and Ulam.

We will not state the theorem in full generality, but only a corollary
sufficient for our purposes. We are given a ``measure'' $m$ on $X$,
which may represent a density of population, of cars, of wealth,
etc. Each subset%
\footnoteE{Not quite every subset is measurable, there are ``exotic''
  subsets, that can not be assigned a measure. However, this
  phenomenon is essentially ``invisible'' in applications and can
  safely be ignored. (Ed.)} %
$P$ of $X$ thus has a certain measure $m(P)$, which may represent the
number of inhabitants in $P$, the number of cars, the total wealth of
the inhabitants of $P$, yearly rainfall at $P$, and so on. We choose
the units of measurement in such a way, that measure of the whole $X$
is equal to 1. Not every measure is suitable for our purposes. We
assume that measure $m$ of some region is zero if and only if its area
is zero. In particular, only two-dimensional regions (not points or
curves) can have positive measure and conversely, the measure of any
``country'' must be positive. Theorem of Oxtoby--Ulam%
\footnoteE{An additional assumption about region $X$ is necessary,
  namely that $X$ is connected (consists of one piece) and
  simply-connected (has no holes inside). (Ed.)} %
implies that there is a \textit{homeomorphism}%
\footnoteE{A continous one-to-one map. (Ed.)} %
from $X$ to the square $f:X\to[0,1]^{2}$ such that
\[
  \area\big(f(P)\big)=m(P)
\]
for all subsets $P$ of $X$. In other words, one can deform the image
of $X$ to a square in such a way, that area of each distorted
region in $X$ will be equal to the measure assigned to it.

The good news is that the theorem is \textit{constructive}, meaning
that one can program a computer to draw the map, as in the examples
above.

A couple of remarks are in order here:
\begin{itemize}
\item The cartogram is not at all unique. One can exploit this
  flexibility to try to impose additional constraints on the map, such
  as best preserving ``something else.'' \textit{Optimal transport theory},%
  \footnoteE{Optimal transport theory (in the Monge formulation)
    concerns, roughly speaking, of finding \textit{optimal way} to
    move one pile of send of a certain shape to another pile of send
    of a different shape but the same volume. Different optimality
    criteria lead to different optimal transportation plans. (Ed.)} %
  which has been growing rapidly in recent years,
  offers an approach that has already been discussed in the archives
  of \textit{Images des maths} and to which we will probably return.

\item Moser’s theorem%
  \footnoteE{The difference between theorems of Moser and of
    Oxtoby--Ulam is that Moser assumes that the density of measure $m$
    is infinitely differentiable. In return, the map garanteed by
    Moser is also infinitely differentiable and also easily
    constructible. (Ed.)} %
  and the practical method it provides --- easy to program on a
  computer --- went completely unnoticed among
  cartographers. Admittedly, Moser himself was likely not particularly
  interested in cartography. It was only recently, in 2004, that
  the theorem was rediscovered by cartographers.%
  \footnoteA{Michael T. Gastner, M. E. J. Newman,
    \textit{Diffusion-based method for producing density equalizing
      maps}, \url{https://arxiv.org/pdf/physics/0401102v1}} %
  Good ideas are often born more then once.
\end{itemize}
}
\end{document}
