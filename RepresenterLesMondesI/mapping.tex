\documentclass{amsart}
\usepackage[utf8]{inputenc}
    
\usepackage{titlesec}
\titleformat{\section}{\normalfont\bf\large}{\relax}{0em}{}
\titleformat{\subsection}{\normalfont\bf}{\relax}{0em}{}

\usepackage{enumitem}
\usepackage{quoting}

\usepackage{xcolor}
\long\def\gray#1{{\color{gray}{#1}}}\let\grey=\gray
\long\def\blue#1{{\color{blue}{#1}}}
\long\def\red#1{{\color{red}{#1}}}

\usepackage{wrapfig}
\usepackage{lpic}

\usepackage{manyfoot}
\DeclareNewFootnote{A}
\DeclareNewFootnote{E}[alph]
\usepackage[hyperfootnotes=false,
    colorlinks,
    linkcolor={blue!80!black},
    citecolor={blue!50!black},
    urlcolor={blue!80!black}]{hyperref}
\def\foothref#1#2{\href{#1}{ #2}\footnoteA{\url{#1}}}

\newcommand{\skipthis}[2][\dots]{\red{[#1]}\gray{#2}}
\newtheorem*{theorem}{Theorem}
\newtheorem{exercise}{Exercise}

\DeclareMathOperator\distance{\mathsf{distance}}

\begin{document}
\centerline{\bf\large Charting the Worlds I%
  \footnoteE{English translation by Rostislav Matveev of the article
    \textit{Repr\'esenter des Mondes}, by
    \'Etienne Ghys, (\red[YEAR?]),
    \url{https://images-des-maths.pages.math.cnrs.fr/%
      freeze/Representer-les-mondes.html}.\\
  }}%

\medskip

\centerline{by \'Etienne Ghys}

\begin{center}
  \includegraphics[width=\textwidth]{Pix/TypusOrbisTerrarum}
\end{center}

Cartography has accompanied mathematics since its very beginnings%
\footnoteE{This is evident from the name of the vast area of
  mathematics --- ``Geometry'', that is ``measuring the Earth''} %
and continually renews its set of problems. Originally, it was a
matter of drawing maps as accurately as possible of continents that
were more or less well known. Above is a map dating from 1570, in
which to the south we see a ``Terra Australis Incognita'' --- which
does not exist.  Today, this mission of the cartographer remains
relevant, but there are other new worlds we would like to represent in
atlases. Cyberspace and the brain are only two examples of these
modern-day terra incognita.

In this series of articles, I would like to present a few selected
pieces of this interaction between cartography and mathematics. The
subject is broad, and I will only touch upon it briefly, but I hope to
show through these examples how the two disciplines mutually enrich
one another.

\section{What is a map?}
Here are two maps.

The one on the left is a reconstruction of Anaximander’s map (around
550 BCE). Described by Herodotus, it is one of the oldest maps
representing the world as a whole. The map on the right is the famous
``Carte du Tendre,'' dating from the 17th century, which depicts the
most mysterious land: that of romantic emotions.
\begin{center}
  \includegraphics[width=0.4\textwidth]{Pix/Anaximandre}%
  \includegraphics[width=0.6\textwidth]{Pix/Carte_du_tendre}
\end{center}

\skipthis[Next par is reduced]
A map is an image. It is a representation of a continent, a country, a
region, etc., on a medium --- most often a sheet of paper --- which
one can, for example, slip into one’s pocket when traveling through
the country in question.
If we denote by $X$ the country and by $Y$ the sheet of paper, the map
is a mapping
\[
  f:X\rightarrow Y
\]

\section{What do we expect from a map?}

\skipthis[Section is reduced]{}
That it be faithful, of course! We hope to gather enough information
in $Y$ to find our way around in $X$. We will see later --- though the
reader can already imagine --- that if it is a matter of making a map of
the New York subway, a sheet of paper will do perfectly well, but if we
want to represent the network of connections among the billions of
neurons that make up our brain, we will probably have to look for
another medium!

Faithfulness in our language means injectivity of a mapping $f$.

\section{Distances}

We also expect a map to be accurate. In general, there is so much
information in $X$ that we would like to represent in $Y$, but $Y$ is
only a small sheet of paper and choices have to be made. Here, I will
focus on a single aspect: that the map preserves \blue{relative}
distances, though there would be many other possibilities.

In ordinary cartography, $X$ is a country on Earth, so two points
$x_1$ and $x_2$ of $X$ are separated by a distance that I will denote
$\distance_{X}(x_{1},x_{2})$.  When $Y$ is a sheet of paper, two
of its points $y_{1}$ and $y_{2}$ are likewise separated by a distance
$\distance_{Y}(y_{1},y_{2})$.

An ideal map would be such that the distance between two points at the
source, in $X$, is exactly the same as the one we measure on the map,
in the target. That is
\[
  \distance_{Y}\big(f(x_{1}),f(x_{2})\big)
  =
  \distance_{X}(x_{1},x_{2})
\]
That said, such a map of France would have to be a thousand kilometers
across, which is not very practical. Of course, we use maps at a
certain scale. For example, if we want one centimeter on the map to
represent one kilometer ``in real life,'' that means that distances are
multiplied in the target (on paper) by $k = 0.00001 = 1/100000$. Thus,
for an ideal map at scale $k$ we would instead require:
\[
  \distance_{Y}\big(f(x_{1}),f(x_{2})\big)
  =
  k\cdot\distance_{X}(x_{1},x_{2})
\]

\section{The ideal does not exist!}

I claim that there is no map of Europe, for instance, that preserves distances exactly.

To prove this, I consider four cities: Athens, Madrid, Paris, and
Oslo. Here is a table giving the distances between them.%
\footnoteA{I had intended to compute these distances myself from the
  latitudes and longitudes of the cities, but I discovered a website
  that does it for me in just a few clicks\dots The calculations
  assume that the Earth is perfectly spherical, which it is not
  quite\dots, but this does not change anything in the argument.}
\medskip

\begin{center}
  \begin{tabular}{|c||c|c|c|c|}
    \hline
    \mbox{}& Athens & Madrid & Paris & Oslo\\
    \hline\hline
    Athens & 0km & 1743km & 2414km & 2612km \\
    \hline
    Madrid & 1743km & 0km & 936.6km & 2224km \\
    \hline
    Paris  & 2414km & 936.6km & 0km & 1351km \\
    \hline
    Oslo   & 2612km & 2224km & 1351km & 0km \\
    \hline
  \end{tabular}
\end{center}
\medskip

Let us try to construct an exact map of Europe, say at scale
$k = 1/100000$ (one centimeter equals one kilometer). To do this, we
must first mark the (image of a) first city, say Athens. This is the
point $A$. Next, we place a second city, say Madrid: the point $M$
must be 17.43 centimetres from $A$. No problem so far\dots We can place $M$ at
any point on a circle, but in any case all these positions are
equivalent, up to rotating the sheet of paper around $A$.

Next, we place Oslo. We must place a point $O$ at 26.12 cm from $A$
and 22.24 cm from $M$. So we must construct two circles, centered at
$A$ and $M$, with radii 26.12 cm and 22.24 cm, and place $O$ at their
intersection. There are two intersection points, and we may choose
either one, since the other is obtained from the first by
symmetry. Let us therefore choose the one that respects the
orientation we are used to, that is ``to the North''. So far so good.

\begin{center}
  \hfill
  \includegraphics[width=0.4\textwidth]{Pix/dessin1}
  \hfill
  \includegraphics[width=0.4\textwidth]{Pix/dessin}
  \hfill\rule{0mm}{1mm}
\end{center}

It remains to place Paris on the map.
\begin{itemize}[leftmargin=4mm,rightmargin=0mm]
\item We know the distance from Paris to Oslo and to Athens: this
  gives us only two possibilities for placing Paris, at $P_1$ or
  $P_1'$.
\item We know the distance from Paris to Oslo and to Madrid: this also
  gives us only two possibilities for placing Paris, at $P_2$ or
  $P_2'$.
\item We know the distance from Paris to Athens and to Madrid: this
  gives us only two possibilities for placing Paris, at $P_3$ or
  $P_3'$.
\end{itemize}

The problem is that these possibilities are not consistent: the points
$P_1$ and $P_1'$ do not coincide with $P_2$ and $P_2'$\dots They are not
very far from one another, but they do not coincide\dots
\medskip

\textbf{It is impossible to draw an exact map of Europe!}
\medskip
\begin{exercise}
  Could one find four cities that, unlike the ones that we considered
  above, can be represented in the plane in such a way as to respect
  the distances exactly (up to a scale)? Can you find all the
  situations in which this is possible?
\end{exercise}

\begin{exercise}[Difficult]
  Given six positive numbers $d_{1,2}$, $d_{1,3}$, $d_{1,4}$,
  $d_{2,3}$, $d_{2,4}$ and $d_{3,4}$, under what
  condition is it possible to find four points in the plane
  $P_1$, $P_2$, $P_3$, $P_4$ such that the pairwise distances
  $\distance(P_{i}, P_{j})$ between them are exactly $d_{i,j}$?
\end{exercise}




\end{document}