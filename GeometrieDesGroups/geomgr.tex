\documentclass{amsart}
\usepackage[utf8]{inputenc}
\usepackage{url}

\usepackage{bm}
\def\1{\bm{1}}

\usepackage{wrapfig}

\usepackage{titlesec}
\titleformat{\section}{\normalfont\bf\large}{\thesection. }{0em}{}
\titleformat{\subsection}{\normalfont\bf}{}{0em}{}

\usepackage{manyfoot}
\DeclareNewFootnote{A}
\DeclareNewFootnote{E}[alph]

\usepackage{xcolor}
\long\def\gray#1{{\color{gray}{#1}}}
\long\def\blue#1{{\color{blue}{#1}}}
\long\def\red#1{{\color{red}{#1}}}

\usepackage{lpic}

\def\sltwo{\ensuremath{\mathrm{SL}(n,\mathbb{Z})}}

\begin{document}
\centerline{\bf A Bit of Geometric Group Theory}
\medskip
\centerline{by  Gilbert Levitt}

\bigskip
\centerline{
  \begin{lpic}[b(28mm),draft,clean]{Pix/escher-circle-limit-III(1)}
    \lbl[t]{28,-2;
      \parbox{0.8\textwidth}{
        \hfill Maurits Cornelis Escher, \textit{Circle Limit
          III}, Print, 1959\hfill\ 
        \vspace{1ex} \\
        \sf Discrete groups appear in every area of mathematics
        --- and even in Escher’s art.  Even if they are defined algebraically,
        we often understand them better by their action on
        geometric objects. More and more often, they are viewed as
        geometric entities in their own right. Their properties are
        especially striking when the curvature is negative.%
      }}
  \end{lpic}
}

\section{A Few Examples of Groups}
Groups we will consider will be generally non-commutative (see
Appendix). We will write group operation multiplicatively, with the
neutral element in group $G$ denoted $\1_{G}$ or simply $\1$. Groups
that will interest us most are finitely generated --- that is
they can be generated by a finite number of elements. Let’s take a
look at a few examples.
\begin{itemize}
\item The \textit{free Abelian group} $\mathbb{Z}^2$, or
  $\mathbb{Z} \times \mathbb{Z}$, or $\mathbb{Z} \oplus \mathbb{Z}$,
  is the set of pairs of integers $(m, n)$, with addition defined by
  $(m, n) + (m', n') = (m + m', n + n')$.  To write it
  multiplicatively, let $a:=(1,0)$, $b:=(0,1)$, and view
  $\mathbb{Z}^2$ as the set of elements $a^m b^n$, equipped with the
  multiplication rule
  $(a^m b^n)(a^{m'} b^{n'}) = a^{m + m'} b^{n + n'}$.  The neutral
  element $a^0 b^0$ is denoted $\1$, and the inverse of $a^m b^n$ is
  $a^{-m} b^{-n}$.
\item Let us consider the group $\mathrm{Aff}(\mathbb{R})$ acting on the real
  line $\mathbb{R}$ by homotheties and translations --- that is, the
  transformations of the form $x \mapsto a\cdot x + b$ with
  $a, b \in \mathbb{R}$ and $a \neq 0$, the product being given by
  composition $(f \circ g)(x) = f(g(x))$.

  This is a ``continuous'' group (a Lie group), but we can consider
  finitely generated subgroups, for example the group $G_1$ generated
  by $t : x \mapsto x + 1$ and $h : x \mapsto 2x$.
  One can deduce that $G_1$ is the set of transformations $\varphi_{mnp}$
  of the form $\varphi_{mnp}(x) = 2^m x + n 2^p$, with
  $m, n, p \in \mathbb{Z}$ (see Appendix).

\item The group $\mathrm{GL}(n, \mathbb{R})$ of invertible (with
  determinant $\neq 0$)
  $n \times n$ matrices with real coefficients is also a Lie group.

  The matrices with integer entries do not form a subgroup, because
  the determinant appears in the denominator when computing the
  inverse of a matrix.
  However, \sltwo, the set of matrices with
  integer entries and determinant $1$, is a subgroup.

  We will consider the group $G_2 \subset \sltwo$
  generated by
  \[
    A=\begin{pmatrix}1&0\\2&1\end{pmatrix},\quad
    B=\begin{pmatrix}1&2\\0&1\end{pmatrix}
  \]
\end{itemize}

\section{Free Groups}

\red{[S: I don't think, that this paragraph is necessary or adds anything.]}
In a vector space $V$ over a field $K$, the \textit{vector subspace}
generated by vectors $\{v_1, \dots, v_k\}$ in $V$ is the set of all
linear combinations
\[
  \sum_{i=1}^{k}\lambda_{i}v_{i},
\]
with $\lambda_{i}\in K$.  The elements $v_1, \dots, v_k$ are
\textit{linearly independent} if any two different linear combinations
represent different elements of $V$, or equivalently, if there is no
relation
\[
  \sum_{i=1}^{k}\lambda_{i}v_{i}=0
\]
with the $\lambda_i$ not all zero.
The subspace generated has then dimension $k$ and is isomorphic to $K^k$.

In a group $G$, the subgroup generated by ${g_1, \dots, g_k}$ is the
set of all elements of $G$ that can be written as a reduced word
$g_{i_{1}}^{n_{i}}\dots g_{i_{p}}^{n_{p}}$, where the $n_j$ are
nonzero integers and $i_j \neq i_{j+1}$.  For example, $a^2$,
$b^{-1}c$, and $c^{-3}a^3b^2acb^{-5}$ are reduced words in $a, b, c$.
Care must be taken not to forget the empty word, denoted $1$, which
represents the identity element $1_G$.  The length $|W|$ of a word $W$
is the total number of letters, taking exponents into account, for
example $|c^{-3}a^{3}b^{2}acb^{-5}|=15$.

We say that elements $g_1, \dots, g_k$ of $G$ are
\textit{independent}%
\footnoteE{Often, in English literature, by \textit{independence} of
  the collection $\{g_1, \dots, g_k\}$ of elements of a group, another, in
  general strictly stronger, condition is meant --- namely, that no
  element is equal to a reduced word in the other elements of the
  collection; thus \textit{free family} seems to be a more
  appropriate term. (Ed.) }
(or form a \textit{free family}) if two different reduced words always
represent two different elements of $G$, or equivalently, if there is
no nontrivial relation $g_{i_1}^{n_1} \cdots g_{i_p}^{n_p} = 1$. For
example, the family $\{g\}$ (consisting of the single element
$g\in G$) is free if and only if there is no nontrivial relation
$g^n = 1$, that is, if $g$ has infinite order.

In the examples above, the families ${a, b} \subset \mathbb{Z}^2$ and
${h, t} \subset G_1$ are not free, because of the relations $ab = ba$
and $hth^{-1} = t^2$.

We will, however, show --- using the so-called ping-pong technique ---
that the matrices $A$ and $B$ are independent in \sltwo.

To this end, let us make \sltwo{} act on
$P = \mathbb{R} \cup {\infty}$ (the real projective line) by
associating to the \sltwo-matrix
\[
  M = \begin{pmatrix} a & b \\ c & d \end{pmatrix}
\]
the \textit{homography}%
\footnoteE{Synonym: \textit{projective transformation}. (Ed.)}
\[
  h_M : x \mapsto \dfrac{a x + b}{c x + d}
\]
with the usual conventions, in particular $h_M(-d/c) = \infty$ and
$h_M(\infty) = a/c$ if $c \neq 0$. The definition is constructed so that
$h_{MN} = h_M h_N$ for all pairs of $\sltwo$-matrices $M$ and $N$.

Let $P_A = (-1, 1)$, and let $P_B$ be the complement of $[-1, 1]$ in
$P$. We have $h_A(x) = x + 2$, and therefore $h_A^n(P_A) \subset P_B$
for all $n \neq 0$. Similarly, $h_B(x) = x/(2x+1)$
and $h_B^n(P_B) \subset P_A$ for $n \neq
0$.%
\footnoteE{Note also, that the inclusions $h_A^n(P_A) \subset P_B$ and
  $h_B^n(P_B) \subset P_A$ are strict. (Ed.)}
Let's now play ping-pong with $P_A$ and $P_B$.

To show that $A$ and $B$ are independent, consider a nontrivial
reduced word, for example $W = B^2 A B^{-3} A^5$. Apply
$h_W = h_B^2 h_A h_B^{-3} h_A^5$ to $P_A$. The element $h_A^5$ sends
it into $P_B$, the element $h_B^{-3}$ sends it back into $P_A$, and so
on, and finally $h_W(P_A)$ is contained in $P_A$, but not equal to
it. This prevents $h_W$ from being the identity, and therefore $W$
from being equal to $1$ in $\mathrm{SL}(2, \mathbb{Z})$.

This reasoning applies to any word $W$ beginning with a power of $B$
and ending with a power of $A$. The other cases are treated similarly:
if $W$ begins and ends with a power of $A$, we have
$h_W \neq \mathrm{id}$ because $h_W(P_A) \subset P_B$; if $W$ ends
with a power of $B$, we apply $h_W$ to $P_B$.

Since $A$ and $B$ form a free family, every element of $G_2$ can be
written uniquely as a reduced word in $A$ and $B$. At this point we
can forget that $A$ and $B$ are matrices and regard $G_2$ as the set
$F(A, B)$ of reduced words in two abstract symbols $A$ and
$B$. Multiplication consists of concatenation and reduction; for
example, $(B^2 A B^{-3} A^5)(A^{-5} B A^4) = B^2 A B^{-2} A^4$, and
the inverse of $B^2 A B^{-3} A^5$ is $A^{-5} B^3 A^{-1} B^{-2}$.

We say that $G_2$ is the free group of rank 2, often denoted
$F_2$. Similarly, we define $F_n$, the free group of rank $n$, for
$n > 2$.

Many groups contain subgroups which are free groups. For example, one
can show that two randomly chosen rotations of the sphere generate a
free group, as do the transformations $x \mapsto x + 1$ and
$x \mapsto x^3$ on $\mathbb{R}$.

The group $F_2$ contains arbitrarily large free families: it is easy
to see, that the infinite family $\{A^n B A^{-n}\}_{n \in \mathbb{N}}$ is
free, because the $B$’s do not cancel when these elements are
multiplied. The free group of rank 2 therefore contains free groups of
any rank, and even free groups that are not finitely generated. The
Nielsen–Schreier theorem guarantees that every subgroup of a free
group is free, that is, it is generated by a free family.

\section{Tits Alternative}

We have already noted that $G_1$ is not free, since its generators
satisfy $h t h^{-1} = t^2$. To find other relations, observe that in
$\mathrm{Aff}(\mathbb{R})$, and therefore in $G_1$, every commutator
$[g_1, g_2] := g_1 g_2 g_1^{-1} g_2^{-1}$ is a translation, and that
two translations commute. Therefore any two commutators commute,
$[g_1, g_2][g_3, g_4] = [g_3, g_4][g_1, g_2]$ for all
$g_1, g_2, g_3, g_4 \in G_1$. This ``universal'' relation expresses that
$G_1$ is \textit{metabelian}, or equivalently, \textit{solvable} of class 2.

More generally, we say that $G$ is solvable of class $\leq p$ if the
subgroup generated by all commutators $[g_1, g_2]$ is solvable of
class $\leq p - 1$, that is, if any $2p$ elements of $G$ satisfy a
certain identity built from iterated commutators. Solvable groups are
those that can be obtained by successive extensions from commutative
groups. The impossibility of solving algebraic equations of degree 5
by radicals is due to the non-solvability of the symmetric group $S_5$.
This is subject of the \textit{Galois theory}.

It is easy to verify that, for any field $K$, the subgroup of
$\mathrm{GL}(n, K)$ consisting of invertible upper triangular matrices
is solvable (of class $n$). The famous Tits alternative (1972) states
that if a finitely generated group $G$ is linear --- that is, isomorphic
to a subgroup of some $\mathrm{GL}(n, K)$ --- then either $G$ contains a
subgroup isomorphic to $F_2$, or a subgroup of finite index in $G$
(see Appendix) is solvable.

In other words, either $G$ contains arbitrarily large free families,
or (up to finite index) the elements of $G$ satisfy a universal
relation. The Tits alternative has been extended to other classes of
groups; for instance, Bestvina, Feighn, and Handel have recently
proved it for subgroups of the group $\mathrm{Out}(F_n)$ of
automorphisms of a finitely generated free group, modulo conjugations.


\section{Relations and Presentations}

If a group $G$ is not free, different reduced words may represent the
same element; we say that such words are equivalent in $G$. The
\textit{word problem} is the problem of determining, by an algorithm,
whether two given words represent the same element. In fact, it
suffices to determine which words are trivial, that is, which
represent the neutral element $\1_G$.

This is easy in $\mathbb{Z}^2$, which is commutative. For instance, it
is immediately clear to us that
$a^{100} b^{100} a^{-100} b^{-100} = \1$. But a machine that could
only apply mechanically the basic relation $ab = ba$ (and, to be
generous, the relations $a^{\pm1} b^{\pm1} = b^{\pm1} a^{\pm1}$) would
find it tedious to show this equality: it would in fact have to move
each of the one hundred $a^{-1}$’s past each $b$, that is, about
$10\,000$ operations for a word of length $400$. In general, the
number of operations required to show that a word of length $n$ in
$\mathbb{Z}^2$ is trivial, in the worst case, is of the order of $n^2$
for large $n$. We say that $\mathbb{Z}^2$ has a quadratic
isoperimetric inequality.

The geometric interpretation is as follows (see Figure 1). Tile the
plane with a grid whose horizontal edges are oriented to the right and
labeled $a$, and whose vertical edges are oriented upward and labeled
$b$. Fix a vertex $E$ of this graph as the origin. A word in $a$ and
$b$ can then be represented as a path starting from $E$; for example,
$a^2 b^{-1} a^{-1} b^{-1}$ moves two units to the right, one down, one
to the left, and one down again.

% \centerline{
  % \begin{lpic}[draft,l(11mm),r(11mm),b(21mm),t(2mm),clean]{Pix/z2(0.4)}
    % \lbl[b]{60,73;\color{blue}$a$}
    % \lbl[b]{80,73;\color{blue}$a$}
    % \lbl[l]{93,60;\color{blue}$b^{-1}$}
    % \lbl[b]{80,53;\color{blue}$a^{-1}$}
    % \lbl[l]{73,40;\color{blue}$b^{-1}$}
    % \lbl[b]{60,33;\color{blue}$a^{-1}$}
    % \lbl[l]{53,20;\color{blue}$b^{-1}$}
    % \lbl[t]{43,9;\color{blue}$a^{-1}$}
    % \lbl[r]{27,20;\color{blue}$b$}
    % \lbl[b]{23,33;\color{blue}$a^{-1}$}
    % \lbl[r]{7,40;\color{blue}$b$}
    % \lbl[b]{40,73;\color{red}$a^{-1}$}
    % \lbl[b]{20,73;\color{red}$a^{-1}$}
    % \lbl[r]{9,60;\color{red}$b^{-1}$}
    % \lbl[tW]{50,64;$E$}
    % \lbl[lW]{16.5,50;$F$}
    
    % \lbl[t]{50,-2;\parbox{0.5\textwidth}{
        % \centerline{\bf Figure 1.}
        % \raggedright The words
        % \blue{$a^2 b^{-1} a^{-1} b^{-1} a^{-1} b^{-1} a^{-1} b a^{-1} b$} and
        % \red{$a^{-2} b^{-1}$} represent the same element of $\mathbb{Z}^2$:
        % the corresponding paths have the same endpoint $F$.}}
  % \end{lpic}
% }

\begin{wrapfigure}{r}{0mm}
  \begin{lpic}[draft,l(8mm),r(4mm),b(21mm),t(-5mm),clean]{Pix/z2(0.4)}
    \lbl[b]{60,73;\color{blue}$a$}
    \lbl[b]{80,73;\color{blue}$a$}
    \lbl[l]{93,60;\color{blue}$b^{-1}$}
    \lbl[b]{80,53;\color{blue}$a^{-1}$}
    \lbl[l]{73,40;\color{blue}$b^{-1}$}
    \lbl[b]{60,33;\color{blue}$a^{-1}$}
    \lbl[l]{53,20;\color{blue}$b^{-1}$}
    \lbl[t]{43,9;\color{blue}$a^{-1}$}
    \lbl[r]{27,20;\color{blue}$b$}
    \lbl[b]{23,33;\color{blue}$a^{-1}$}
    \lbl[r]{7,40;\color{blue}$b$}
    \lbl[b]{40,73;\color{red}$a^{-1}$}
    \lbl[b]{20,73;\color{red}$a^{-1}$}
    \lbl[r]{9,60;\color{red}$b^{-1}$}
    \lbl[tW]{50,64;$E$}
    \lbl[lW]{16.5,50;$F$}
    
    \lbl[t]{50,-2;\parbox{0.4\textwidth}{\sf\small
        \raggedright The words
        \blue{$a^2 b^{-1} a^{-1} b^{-1} a^{-1} b^{-1} a^{-1} b a^{-1} b$} and
        \red{$a^{-2} b^{-1}$} represent the same element of $\mathbb{Z}^2$:
        the corresponding paths have the same endpoint $F$.}}
  \end{lpic}
\end{wrapfigure}

We note that two words are equivalent if and only if their associated
paths have the same endpoint; for example,
$a^2 b^{-1} a^{-1} b^{-1} a^{-1} b^{-1} a^{-1} b a^{-1} b$ is
equivalent to $a^{-2} b^{-1}$. In particular, the vertices of the
graph correspond to the elements of $\mathbb{Z}^2$, and a word is
trivial if and only if the associated path is a loop (it closes back
at $E$). Thus, $a^{100} b^{100} a^{-100} b^{-100}$ represents the
boundary of a square of side $100$. Applying the relation
$a^{\pm1} b^{\pm1} = b^{\pm1} a^{\pm1}$ amounts to having the loop
cross one cell of the grid, and $10{,}000$ is simply the area of the
square.

The exponent $2$ obtained above is thus the one that expresses the
area of a square as a function of its side. We can see the analogy
with the classical isoperimetric inequality, which bounds the area
enclosed by a plane curve by the square of its length (divided by
$4\pi$, though that detail is not important here).

Returning to algebra, we will now explain how to solve the word
problem in $G_1$ using only the relation $h t h^{-1} t^{-2} =
1$. Thanks to the equations $h t^{\pm1} = t^{\pm2} h$ and
$t^{\pm1} h^{-1} = h^{-1} t^{\pm2}$, one can, in any word, move all
positive powers of $h$ to the right of the word and all negative
powers to the left. In other words, any word $W$ is equivalent in
$G_1$ to a word of the form $h^{-m} t^n h^p$ with $m, p \ge 0$. Such a
word represents the transformation $x \mapsto 2^{p - m} x + 2^{-m} n$,
which is the identity if and only if $n = 0$ and $p = m$, that is, if
the word is empty. Therefore, $W = 1$ in $G_1$ if and only if the word
$h^{-m} t^n h^p$ associated to $W$ is the empty word: the word problem
is solved.

This reasoning actually shows that all relations satisfied by $h$ and
$t$ can be formally deduced from the relation $h t h^{-1} t^{-2} =
1$. We say that $G_1$ is presented by the generators $h$ and $t$
subject to the relation $h t h^{-1} t^{-2} = 1$.

In general, we say that
$G = \langle g_1, \dots, g_k \mid r_1, \dots, r_\ell \rangle$, where
the $r_j$ are words in the $g_i$, is a presentation of $G$ if $G$ is
generated by elements $g_i$ satisfying the relations $r_j = 1$, and if
every relation among the $g_i$ can be formally deduced from the
relations $r_j = 1$ (more precisely, $G$ is isomorphic to the quotient
of the free group $F(g_1, \dots, g_k)$ by the subgroup generated by
all products of conjugates of the $r_j$ and their inverses).

Fix an integer $m$, and now ask a machine to prove the relation
$[h^m t h^{-m}, t] = 1$ from $h t h^{-1} t^{-2} = 1$. This is easy for
us, since we can see that $h^m t h^{-m} = t^{2^m}$. But for the
machine, the number of operations will be on the order of $2^m$, that
is, an exponential function of the length of $[h^m t h^{-m}, t]$
(equal to $4m + 4$). Since the words $[h^m t h^{-m}, t]$ are
representative of the general case, $G_1$ satisfies an exponential
isoperimetric inequality.


\section{The Dehn Function}

Given a finite presentation
$G = \langle g_1, \dots, g_k \mid r_1, \dots, r_\ell \rangle$, we
define the Dehn function $\varphi(n)$, whose growth determines the
isoperimetric inequality satisfied by $G$.  A replacement such as
$a^{\pm1} b^{\pm1} \mapsto b^{\pm1} a^{\pm1}$ or
$h t^{\pm1} \mapsto t^{\pm2} h$ amounts to multiplying the word by a
conjugate of some $(r_j)^{\pm1}$, and a word $W$ is trivial in $G$ if
and only if, in the free group $F(g_1, \dots, g_k)$, it can be written
as $W = \prod_{m=1}^{s} u_m (r_{j_m})^{\pm1} u_m^{-1}$.  For each
trivial word $W$, we consider the smallest possible $s$, and
$\varphi(n)$ is the maximum of these $s$ for all trivial words of
length $\le n$.

The Dehn function depends on the presentation, but the manner of its
growth (quadratic, exponential, etc.) depends only on $G$. We have
said that $\varphi$ is quadratic for $\mathbb{Z}^2$ and exponential
for $G_1$; here is an example of a linear $\varphi$.

Let $G_3$ be the group with presentation
$\langle a, b, c, d \mid d^{-1} d a b a^{-1} b^{-1} c d c^{-1} d^{-1}
\rangle$ (the fundamental group of the closed orientable surface of
genus 2).  Given a word $W$ in $a, b, c, d$, we can shorten it if it
contains more than half of the relation (or its inverse), up to cyclic
permutation.  For instance, we may replace $aba^{-1}b^{-1}c$ by
$d c d^{-1}$, or $d^{-1} c^{-1} b a b^{-1}$ by $c^{-1} d^{-1} a$, or
$d c^{-1} d^{-1} a b$ by $c^{-1} b a$, and so on. We then reduce the
resulting word (if possible) and repeat the process as long as
possible.

Dehn showed (around 1910) that in $G_3$, this procedure—called Dehn’s
algorithm—solves the word problem: $W$ represents 1 if, and more
importantly only if, the algorithm terminates with the empty
word. Since the length of the word decreases at each step, the number
of operations is bounded by the length of the word; thus $G_3$
satisfies a linear isoperimetric inequality.

Groups in which the word problem can be solved by Dehn’s algorithm
(shortening the word whenever it contains more than half of a
relation) have a Dehn function that is at most linear. Conversely, one
can show that a group with Dehn function at most linear admits a
presentation for which Dehn’s algorithm applies. These groups are
precisely the hyperbolic groups defined by Gromov around 1985; we will
discuss their geometric aspects below.

If $G$ is not hyperbolic, its Dehn function is at least quadratic. On
the other hand, there is no “gap” beyond the exponent 2: N. Brady and
M. Bridson have recently shown that the set of exponents $\alpha$ for
which there exists a group whose Dehn function is equivalent to
$n^\alpha$ is dense in $[2, +\infty[$ (note that the set of
isomorphism classes of finitely presented groups is countable, and
therefore so is the set of these $\alpha$).

Knowing the Dehn function of a finitely presented group explicitly
allows one to solve the word problem algorithmically in that group: to
determine whether a word $W$ of length $n$ is trivial, it suffices to
compare it with all expressions
$\prod_{m=1}^{s} u_m (r_{j_m})^{\pm1} u_m^{-1}$ with
$s \le \varphi(n)$, of which there are only finitely many (the lengths
of the words $u_m$ can be bounded a priori). Conversely, an algorithm
that solves the word problem makes it possible to compute $\varphi$.

It is known that there exist finitely presented groups in which the
word problem cannot be solved algorithmically, because the Dehn
function is non-recursive: it grows so fast that no algorithm can
compute it. Thus, in complete generality, nothing can be said about a
group given by generators and relations—not even whether the group is
trivial or not. However, in most cases, any algebraic or geometric
information about $G$, even minimal, allows one to analyze it.


\section{The Hyperbolic Plane $\mathbb{H}^2$}

Let us look at the group $G_3$ from a geometric point of view (as Dehn
did). Let us try to construct a graph as we did for
$\mathbb{Z}^2$. The graph to consider is no longer of degree 4, but of
degree 8: from each vertex emerge 4 edges labeled $a, b, c, d$, and 4
edges arrive there. The cells of the grid are octagons, corresponding
to the relation $a b a^{-1} b^{-1} c d c^{-1} d^{-1}$.

We can try to draw this graph, but we quickly run out of space to fit
8 edges at each vertex: the Euclidean plane cannot be tiled by regular
octagons. This graph must actually be drawn not in the Euclidean
plane, but in the hyperbolic plane $\mathbb{H}^2$.

Imagine a circular swimming pool (more mathematically, the open unit
disk $D$ in the plane) filled with a viscous fluid that becomes denser
as one approaches the edge: the viscosity coefficient is proportional
to $\frac{1}{1 - r^2}$, where $r$ is the Euclidean distance from the
center of the disk. The (hyperbolic!) distance between two points
$x, y$ in $D$ is defined as the time it would take a swimmer to go
from $x$ to $y$.

There always exists a shortest path from $x$ to $y$ (called a
geodesic), but it does not appear straight to us: it bends toward the
center of the disk to allow a faster route, just as an airplane climbs
in altitude to reduce air resistance. The geodesics are in fact arcs
contained in circles perpendicular to the boundary of $D$, as well as
the diameters (note that the boundary of $D$ is “at infinity”; it
cannot be reached in finite time).

Like the Euclidean plane, the hyperbolic plane is a homogeneous metric
space: any point can be sent to any other by an isometry (in
particular, the “center” of $D$ plays no special role; any homography
of the complex plane that maps $D$ onto itself induces an
isometry). But it has negative curvature, whereas the Euclidean plane
has zero curvature, and the sphere has positive curvature.

\begin{wrapfigure}{r}{0mm}
  \begin{lpic}[b(23mm)]{Pix/escher-circle-limit-III(0.8)}
    \lbl[t]{28,-2; \parbox{0.3\textwidth}{\sf\small M. C. Escher used
        tilings of $\mathbb{H}^2$; for example, \textit{Circle Limit III}
        evokes a tiling by regular triangles and quadrilaterals,
        separated by white geodesic lines.  }}
  \end{lpic}
\end{wrapfigure}

To study $G_3$, we tile $\mathbb{H}^2$ with regular octagons whose
sides are geodesic segments of equal length and whose angles are
$2\pi / 8$ (that is, $45^\circ$), in such a way that eight octagons
meet at each vertex. The graph associated with such a tiling is
precisely the “grid” we were seeking for $G_3$.

The Dehn function of $G_3$ is therefore linear, since $\mathbb{H}^2$
satisfies a linear isoperimetric inequality: the area bounded by a
curve can be bounded by a linear function of its length. For example,
a disk of radius $R$ has area $2\pi \sinh R$, comparable to its
perimeter $2\pi (\cosh R - 1)$. (It’s better to do jigsaw puzzles in
the hyperbolic plane: once you’ve placed the border, you’ve already
set a non-negligible proportion of the pieces.)

Elementary geometry in $\mathbb{H}^2$ holds other surprises. The
Euclidean parallel postulate is not true, and the sum of the angles of
a triangle is not equal to $\pi$ (i.e. $180^\circ$): it is equal to
$\pi$ minus the area of the triangle (in particular, the area of a
triangle is at most $\pi$).

Another fundamental property of $\mathbb{H}^2$ is the thinness of
triangles: there exists a constant $\delta$ (equal to
$\log(2 - \sqrt{2} + 1)$) such that every point on one side of a
geodesic triangle lies at distance at most $\delta$ from some point on
one of the other two sides. This property is called
$\delta$-hyperbolicity, or simply hyperbolicity.

\section{Hyperbolic Groups and Quasi-Isometries}

We have seen that the group $\mathbb{Z}^2$ “resembles” the Euclidean
plane, while $G_3$ “resembles” the hyperbolic plane. Following
M. Gromov, we formalize this idea by viewing a group $G$, equipped
with a finite generating set $S$, as a metric space: the distance
between two elements $g$ and $h$ of $G$ is the minimal length of a
word (written with the elements of $S$) representing $g^{-1}h$.

This “discrete” space is more easily visualized as the set of vertices
of the Cayley graph of $G$: we place an edge between two vertices
$g, h$ if $h$ is obtained from $g$ by right-multiplication by an
element of $S$, and we declare that each edge is a segment of length
1. The distance between two points is then the length of a shortest
path connecting them (such a path is again called a geodesic).

The Cayley graph of the free group $G_2$ is a tree (it has no
loops). That of $\mathbb{Z}^2$ is the grid used earlier, with the
so-called Manhattan (or taxicab) distance (note that in general there
may be several geodesics between two given points). This distance is
not the Euclidean one (which corresponds to “as the crow flies”), but
it is comparable: the ratio of the two distances lies between two
strictly positive constants (here 1 and $\sqrt{2}$). Similarly, the
Cayley graph of $G_3$ is formed by the geodesics bounding the octagons
in the tiling of $\mathbb{H}^2$ mentioned above, with a distance
comparable to the hyperbolic distance.

A group $G$ is called hyperbolic if there exists a constant $\delta$
such that its Cayley graph is $\delta$-hyperbolic, meaning that the
triangles in the Cayley graph, like those in $\mathbb{H}^2$, are
thin. Thus $G_2$ and $G_3$ are hyperbolic, while $\mathbb{Z}^2$ is not
(and neither is $G_1$).

In our examples, we have always chosen the most natural, simplest
generating set. But a finitely generated group has infinitely many
generating sets, and hence infinitely many distinct Cayley graphs, so
it is not immediately clear that all of them are hyperbolic if one of
them is.

In fact, all these graphs resemble one another, just as they resemble
the discrete space $G$ considered earlier—just as the Cayley graph of
$\mathbb{Z}^2$ resembles the Euclidean plane and that of $G_3$
resembles $\mathbb{H}^2$. This resemblance is to be understood in the
sense of quasi-isometry.

Two metric spaces $X, Y$ are quasi-isometric if there exists a map
$f : X \to Y$ and a constant $\lambda > 1$ such that $f$ does not
distort distances too much for sufficiently distant points (the ratio
between $d_Y(f(x), f(x'))$ and $d_X(x, x')$ lies between
$\frac{1}{\lambda}$ and $\lambda$ whenever $d_X(x, x') > \lambda$),
and $f$ is almost surjective (every ball of radius $\lambda$ in $Y$
contains a point of the image).

Every bounded space is quasi-isometric to a point; quasi-isometry is
meant to capture asymptotic properties of spaces—that is, properties
“at infinity.” Two (geodesic) spaces that are quasi-isometric are
simultaneously hyperbolic or not, which justifies the definition of a
hyperbolic group given above.

Thus, to every finitely generated group one can associate a
well-defined metric space, up to quasi-isometry. One can therefore
speak of groups that are quasi-isometric to each other. Many algebraic
or geometric properties of groups are invariant under quasi-isometry:
for example, being finite, finitely presented, containing a
commutative subgroup of finite index, being hyperbolic, or having a
Dehn function of a given growth type. In general, one seeks to
classify groups up to quasi-isometry.

Let us mention a recent rigidity result due to Farb and Mosher. For an
integer $n \ge 2$, let $H_n$ be the subgroup of
$\mathrm{Aff}(\mathbb{R})$ generated by $x \mapsto x + 1$ and
$x \mapsto n x$ (so $H_2$ is the $G_1$ studied above). If $n$ is a
power of $m$, then $H_n$ is a subgroup of finite index in $H_m$, hence
quasi-isometric to it. Conversely, Farb and Mosher have shown that
$H_n$ and $H_m$ are quasi-isometric if and only if $n$ and $m$ are
powers of the same integer, and (essentially) that any finitely
generated group quasi-isometric to some $H_n$ contains a finite-index
subgroup isomorphic to $H_{n^p}$.

\section{Literature}
\noindent
\textbf{Steve Gersten}, \textit{Introduction to hyperbolic and automatic groups}, Summer School in Group Theory in Banff, 1996, 45—70, CRM Proc. Lecture Notes, 17, Amer. Math. Soc., Providence, RI, 1999.
\medskip

\noindent
\textbf{Mikhail Gromov},  \textit{Hyperbolic groups.} In
\textit{Essays in group theory} (pp. 75-263). New York, NY: Springer
New York. (1987) 
\medskip

\noindent
\textbf{Etienne Ghys, Pierre de la Harpe} (editors), \textit{Sur les
  groupes hyperboliques d’apr\`es Mikhael Gromov}, Progress in
Mathematics 83, Birhäuser.
\medskip

\noindent
\textbf{Andrew Glass}, \textit{The ubiquity of free groups},
Math. Intelligencer 14 (1992), no. 3, 54—57.
\medskip

\noindent
\textbf{Alain Valette}, \textit{Quelques coups de projecteurs sur les
  travaux de Jacques Tits}, Gazette des Mathématiciens No. 61, (1994),
61—79.

\section{Appendix}
\subsection{Groups}
A group is a set in which one can multiply and invert elements. The
product must be associative: we have $(xy)z = x(yz)$, and we simply
write $xyz$. However, the product is not necessarily commutative: we
may have $xy \ne yx$. The inverse of $x$ is denoted $x^{-1}$. It
satisfies $x x^{-1} = x^{-1} x = 1$, where $1$ denotes the identity
element, characterized by $1x = x1 = x$.

For example, the set of all permutations of an $n$-element set is a
finite group, the symmetric group $S_n$; the product of two
permutations $\sigma$ and $\tau$ is the composed permutation, defined
by $(\sigma \circ \tau)(i) = \sigma(\tau(i))$. The set of invertible
$n \times n$ matrices with real entries (that is, with determinant
$\ne 0$) forms a group under matrix multiplication.

If $X$ is any set endowed with some structure, the invertible
transformations that preserve this structure form a group (the product
being, as always, composition—applying the transformations
successively). For example, if $X$ is the Euclidean plane (with its
usual distance), we obtain the group of isometries, which includes in
particular rotations, translations, and orthogonal reflections with
respect to lines.
\subsection{Subgroups}
A subset $A$ of a group $G$ is a subgroup if it is itself a group
under the product of $G$. This holds if and only if $ab$ and $a^{-1}$
are in $A$ whenever $a$ and $b$ are in $A$. If $A$ is not a subgroup,
there exists a smallest subgroup containing $A$; this is the subgroup
generated by $A$. When the subgroup generated by $A$ is the whole of
$G$, we say that $A$ generates $G$. The group $G$ is said to be
finitely generated if it can be generated by a finite subset.

Let us show, for example, that the subgroup $G_1$ generated by
$t : x \mapsto x + 1$ and $h : x \mapsto 2x$ in the affine group
$\mathrm{Aff}(\mathbb{R})$ of transformations of $\mathbb{R}$ consists
of the $\varphi_{mnp}$ of the form $\varphi_{mnp}(x) = 2^m x + n 2^p$,
with $m, n, p \in \mathbb{Z}$. The formulas
$(\varphi_{mnp} \varphi_{m'n'p'})(x) =
\varphi_{mnp}(\varphi_{m'n'p'}(x)) = 2^{m+m'} x + 2^{m+p} n' + 2^{p'}
n 2^{p+p'} = \varphi_{m''n''p''}(x)$ with $m'' = m + m'$,
$n'' = 2^{m+p} n' + 2^{p'} n$, $p'' = p + p'$, and
$\varphi_{mnp}^{-1} = \varphi_{m'n'p'}$ where $m' = -m$, $n' = -n$,
and $p' = m + p$, show that the set of $\varphi_{mnp}$ is a
subgroup. It contains $h$ and $t$, and it is the smallest such
subgroup because $\varphi_{mnp} = h^{-p} t^n h^{m+p}$ is contained in
every subgroup that contains $h$ and $t$.

If $G$ is a finite group and $A$ is a subgroup, the cardinality of $A$
divides that of $G$. The quotient is the number of cosets of $G$
modulo $A$, called the index of $A$. This number is still defined when
$G$ is infinite, but it may itself be infinite. If there exists an
integer $N$ such that whenever $g_0, g_1, \dots, g_N \in G$, one can
find $i \ne j$ with $g_i g_j^{-1} \in A$, then the smallest such
integer $N$ is the index of $A$. Otherwise, the index of $A$ is
infinite. For example, the set of integers that are multiples of $d$
forms a subgroup of $\mathbb{Z}$ that has index $d$ for $d > 0$,
whereas $A = {0}$ has infinite index.
\end{document}