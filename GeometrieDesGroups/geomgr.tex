\documentclass{amsart}
\usepackage[utf8]{inputenc}
\usepackage{url}
\usepackage{bm}
\def\1{\bm{1}}
%%% Comment out to get rid of links at urls in the footnotes 
\usepackage[hyperfootnotes=false]{hyperref}



\usepackage{titlesec}
\titleformat{\section}{\normalfont\bf\large}{\thesection. }{0em}{}

\usepackage{manyfoot}
\DeclareNewFootnote{A}
\DeclareNewFootnote{E}[alph]

\usepackage{xcolor}
\long\def\gray#1{{\color{gray}{#1}}}
\long\def\blue#1{{\color{blue}{#1}}}
\long\def\red#1{{\color{red}{#1}}}

\usepackage{lpic}

% \let\d=\undefined
% \DeclareMathOperator{\d}{d}
\def\sltwo{\ensuremath{\mathrm{SL}(n,\mathbb{Z})}}


\newtheorem*{definition}{Definition}
\newtheorem*{theorem}{Theorem}


\begin{document}
\centerline{\bf A Bit of Geometric Group Theory}
\medskip
\centerline{by  Gilbert Levitt}

\bigskip
\centerline{
  \begin{lpic}[b(28mm),draft,clean]{escher-circle-limit-III(1)}
    \lbl[t]{28,-2;
      \parbox{0.8\textwidth}{
        \hfill Maurits Cornelis Escher, \textit{Circle Limit
          III}, Print, 1959\hfill\ 
        \vspace{1ex} \\
        \sf Discrete groups appear in every area of mathematics
        --- and even in Escher’s art.  Even if they are defined algebraically,
        we often understand them better by their action on
        geometric objects. More and more often, they are viewed as
        geometric entities in their own right. Their properties are
        especially striking when the curvature is negative.%
      }}
  \end{lpic}
}

\section{A Few Examples of Groups}
We consider a group $G$, generally non-commutative. We will write it
multiplicatively, with the neutral element denoted $\1_{G}$ or simply
$\1$. Groups that will interest us most are finitely generated
---that is, that is they can be generated by a finite number of
elements. Let’s take a look at a few examples.
\begin{itemize}
\item The \textit{free abelian group} $\mathbb{Z}^2$, or
  $\mathbb{Z} \times \mathbb{Z}$, or $\mathbb{Z} \oplus \mathbb{Z}$,
  is the set of pairs of integers $(m, n)$, with addition defined by
  $(m, n) + (m', n') = (m + m', n + n')$.  To write it
  multiplicatively, let $a = (1, 0)$, $b = (0, 1)$, and view
  $\mathbb{Z}^2$ as the set of elements $a^m b^n$, equipped with the
  multiplication rule
  $(a^m b^n)(a^{m'} b^{n'}) = a^{m + m'} b^{n + n'}$.  The neutral
  element $a^0 b^0$ is denoted $\1$, and the inverse of $a^m b^n$ is
  $a^{-m} b^{-n}$.
\item Let us consider the group $\mathrm{Aff}(\mathbb{R})$ acting on the real
  line $\mathbb{R}$ by homotheties and translations --- that is, the
  transformations of the form $x \mapsto a\cdot x + b$ with
  $a, b \in \mathbb{R}$ and $a \neq 0$, the product being given by
  composition $(f \circ g)(x) = f(g(x))$.

  This is a ``continuous'' group (a Lie group), but we can consider
  finitely generated subgroups, for example the group $G_1$ generated
  by $t : x \mapsto x + 1$ and $h : x \mapsto 2x$.
  One can deduce that $G_1$ is the set of transformations $\varphi_{mnp}$
  of the form $\varphi_{mnp}(x) = 2^m x + n 2^p$, with
  $m, n, p \in \mathbb{Z}$ (see box).

\item The group $\mathrm{GL}(n, \mathbb{R})$ of invertible (with
  determinant $\neq 0$)
  $n \times n$ matrices with real coefficients is also a Lie group.

  The matrices with integer entries do not form a subgroup, because
  the determinant appears in the denominator when computing the
  inverse of a matrix.
  However, \sltwo, the set of matrices with
  integer entries and determinant $1$, is a subgroup.

  We will consider the group $G_2 \subset \sltwo$
  generated by
  \[
    A=\begin{pmatrix}1&0\\2&1\end{pmatrix},\quad
    B=\begin{pmatrix}1&2\\0&1\end{pmatrix}
  \]
\end{itemize}

\section{Free Groups}

In a vector space $V$ over a field $K$, the \textit{vector subspace} generated
by vectors $\{v_1, \dots, v_k\}$ in $V$ is the set of all linear combinations
\[
  \sum_{i=1}^{k}\lambda_{i}v_{i},
\]
with $\lambda_{i}\in K$.  The elements $v_1, \dots, v_k$ are
\textit{linearly independent} if any two different linear combinations
represent different elements of $V$, or equivalently, if there is no
relation
\[
  \sum_{i=1}^{k}\lambda_{i}v_{i}=0
\]
with the $\lambda_i$ not all zero.
The subspace generated has then dimension $k$ and is isomorphic to $K^k$.

In a group $G$, the subgroup generated by ${g_1, \dots, g_k}$ is the
set of all elements of $G$ that can be written as a reduced word
$g_{i_{1}}^{n_{i}}\dots g_{i_{p}}^{n_{p}}$, where the $n_j$ are
nonzero integers and $i_j \neq i_{j+1}$.  For example, $a^2$,
$b^{-1}c$, and $c^{-3}a^3b^2acb^{-5}$ are reduced words in $a, b, c$.
Care must be taken not to forget the empty word, denoted $1$, which
represents the identity element $1_G$.  The length $|W|$ of a word $W$
is the total number of letters, taking exponents into account, for
example $|c^{-3}a^{3}b^{2}acb^{-5}|=15$.

We say that elements $g_1, \dots, g_k$ of $G$ are
\textit{independent}%
\footnoteE{Often in the literature by indpendence of the collection
  $g_1, \dots, g_k$ of elements of a group, another, in general
  strictly stronger, condition is meant, namely that no element is
  equal to a reduced word in other elements of the collection.}  (or
form a \textit{free family}) if two different reduced words always
represent two different elements of $G$, or equivalently, if there is
no nontrivial relation $g_{i_1}^{n_1} \cdots g_{i_p}^{n_p} = 1$. Thus,
the family ${g}$ (consisting of the single element $g$) is free if and
only if there is no nontrivial relation $g^n = 1$, that is, if $g$ has
infinite order.

In the examples above, the families ${a, b} \subset \mathbb{Z}^2$ and
${h, t} \subset G_1$ are not free, because of the relations $ab = ba$
and $hth^{-1} = t^2$.

We will, however, show --- using the so-called ping-pong technique ---
that the matrices $A$ and $B$ are independent in \sltwo.

To this end, let us make \sltwo{} act on
$P = \mathbb{R} \cup {\infty}$ (the real projective line) by
associating to the \sltwo-matrix
\[
  M = \begin{pmatrix} a & b \\ c & d \end{pmatrix}
\]
the \textit{homography} (or \textit{projective transformation})
\[
  h_M : x \mapsto \dfrac{a x + b}{c x + d}
\]
with the usual conventions, in particular $h_M(-d/c) = \infty$ and
$h_M(\infty) = a/c$ if $c \neq 0$. The definition is made so that
$h_{MN} = h_M h_N$ for all pairs of $\sltwo$-matrices $M$ and $N$.

Let $P_A = (-1, 1)$, and let $P_B$ be the complement of $[-1, 1]$ in
$P$. We have $h_A(x) = x + 2$, and therefore $h_A^n(P_A) \subset P_B$
for all $n \neq 0$. Similarly, $h_B(x) = x/(2x+1)$
and $h_B^n(P_B) \subset P_A$ for $n \neq
0$.%
\footnoteE{Note also, that the inclusions $h_A^n(P_A) \subset P_B$ and
  $h_B^n(P_B) \subset P_A$ are strict.}
Let's now play ping-pong with $P_A$ and $P_B$.

To show that $A$ and $B$ are independent, consider a nontrivial
reduced word, for example $W = B^2 A B^{-3} A^5$. Apply
$h_W = h_B^2 h_A h_B^{-3} h_A^5$ to $P_A$. The element $h_A^5$ sends
it into $P_B$, the element $h_B^{-3}$ sends it back into $P_A$, and so
on, and finally $h_W(P_A)$ is contained in $P_A$ but not equal to
it. This prevents $h_W$ from being the identity, and therefore $W$
from being equal to $1$ in $\mathrm{SL}(2, \mathbb{Z})$.

This reasoning applies to any word $W$ beginning with a power of $B$
and ending with a power of $A$. The other cases are treated similarly:
if $W$ begins and ends with a power of $A$, we have
$h_W \neq \mathrm{id}$ because $h_W(P_A) \subset P_B$; if $W$ ends
with a power of $B$, we apply $h_W$ to $P_B$.

Since $A$ and $B$ are independent, every element of $G_2$ can be
written uniquely as a reduced word in $A$ and $B$. At this point we
can forget that $A$ and $B$ are matrices and regard $G_2$ as the set
$F(A, B)$ of reduced words in two abstract symbols $A$ and
$B$. Multiplication consists of concatenation and reduction; for
example, $(B^2 A B^{-3} A^5)(A^{-5} B A^4) = B^2 A B^{-2} A^4$, and
the inverse of $B^2 A B^{-3} A^5$ is $A^{-5} B^3 A^{-1} B^{-2}$.

We say that $G_2$ is the free group of rank 2, often denoted
$F_2$. Similarly, we define $F_n$, the free group of rank $n$, for
$n > 2$.

Many groups contain subgroups which are free groups. For example, one
can show that two randomly chosen rotations of the sphere generate a
free group, as do the transformations $x \mapsto x + 1$ and
$x \mapsto x^3$ on $\mathbb{R}$.

The group $F_2$ contains arbitrarily large free families: it is easy
to see that the infinite family ${A^n B A^{-n}}_{n \in \mathbb{N}}$ is
free, because the $B$’s do not cancel when these elements are
multiplied. The free group of rank 2 therefore contains free groups of
any rank, and even groups that are not finitely generated. The
Nielsen–Schreier theorem guarantees that every subgroup of a free
group is free, that is, it is generated by a free family.

\end{document}