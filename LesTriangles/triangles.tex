\documentclass{amsart}
\usepackage[utf8]{inputenc}

\usepackage{manyfoot}
\DeclareNewFootnote{A}
\DeclareNewFootnote{E}[alph]
\def\foothref#1#2{\href{#1}{ #2}\footnoteA{\url{#1}}}

\usepackage[hyperfootnotes=false,
    colorlinks,
    linkcolor={blue!80!black},
    citecolor={blue!50!black},
    urlcolor={blue!80!black}]{hyperref}

\usepackage{titlesec}
\titleformat{\section}{\normalfont\bf\large}{\relax}{0em}{}

\usepackage{xcolor}
\long\def\gray#1{{\color{gray}{#1}}}
\long\def\blue#1{{\color{blue}{#1}}}
\long\def\red#1{{\color{red}{#1}}}


\usepackage{wrapfig}
\usepackage{lpic}

\usepackage{enumitem}
\usepackage{quoting}

\let\d=\undefined
\DeclareMathOperator{\d}{d}

\newtheorem*{definition}{Definition}
\newtheorem*{theorem}{Theorem}


\begin{document}
\centerline{\bf Triangles after Euclid, Gauss and Gromov%
  \footnoteE{English translation by Rostislav Matveev of the article
    \textit{Les triangles d’Euclide, de Gauss et de Gromov}, by
    \'Etienne Ghys, (2009),
    \url{https://images-archive.math.cnrs.fr/%
      Les-triangles-d-Euclide-de-Gauss-et-de-Gromov.html}.\\
  }}%
\centerline{A small fragment of Misha Gromov's work}
\medskip
\centerline{by \'Etienne Ghys}

\bigskip
\centerline{\begin{lpic}{Pix/h-triangle(0.3)}
  \end{lpic}
}

\section{Euclid}

For centuries, geometry was Euclid’s geometry --- the one we learn in
school, with its right, isosceles and equilateral triangles and its
theorems of Pythagoras and Thales; the geometry of ``the world in
which we live.'' Euclid established its foundations in the third
century BCE in the book --- a landmark for mathematicians, titled \textit{The
Elements}.%
\footnoteA{Euclid, \textit{The Elements}}$^{,}$%
\footnoteE{English translation available
  for download, for example, at
  \url{https://archive.org/details/euclid\_heath_2nd_ed/1_euclid_heath_2nd_ed/}
(Ed.)}
For more than twenty centuries this book stood at the heart of
mathematics, so definitive did it seem.

Yet one of Euclid’s axioms --- a statement he asked readers to accept
at the beginning of his book without proof --- left a lingering unease:
``Through a point chosen outside a given line, it is possible to draw
exactly one line parallel to the given one.'' Generations of mathematicians
tried to deduce it from the other, seemingly more natural,
axioms. Many believed they had succeeded; many were mistaken.

\section{Non-Euclidean Geometry}

At the beginning of the nineteenth century at least three
mathematicians had the same idea almost simultaneously: there must
exist a non-Euclidean geometry. These were Bolyai, Lobachevsky, and
Gauss. All three broke with a long-held dogma --- they proposed an
entirely new conception of space. It was a genuine mathematical
revolution, comparable to the Copernican revolution of the fifteenth
century.

Of course, it would take very lengthy historical explanations to do
this justice, but that is not the purpose of this article. One of the
central questions concerns existence: what does it mean for another
geometry to exist besides Euclidean? Several geometries? (I note in
passing that my spell-checker refuses to put the word “geometry” in
the plural!) After all, we live in only one space, not two\dots{}
Isn’t the purpose of geometry to describe the relationships between
distances of points in \textit{our} world?

Indeed, this was one of the first times mathematicians needed the
courage to assert that one can (and should) work with spaces that may be
entirely imaginary, with no obvious connection to practical
problems. The links between mathematics and physics --- which, in
principle, deals with what happens in our real world --- have always
been complex.

In any case, at the beginning of the nineteenth century, a few
visionaries began studying a ``new geometry'' in which Euclid’s
parallel postulate does not hold. They proved theorems that differ
from Euclid’s. For example, right triangles still exist, but
Pythagorean theorem is different\dots{} Let there be no mistake: this
does not mean that the Pythagorean theorem we learn in school is
false; it means that it is true in Euclidean space, and that a
different theorem holds in non-Euclidean space --- there is no
contradiction here.  This ``new geometry'' is sometimes called
non-Euclidean, sometimes hyperbolic, and sometimes the
Bolyai–Lobachevskian geometry\dots{}

But all this still seemed somewhat suspicious, and its very existence
was called into question. Things gradually became clearer throughout
the nineteenth century. As always, it took time and the efforts of
many mathematicians, but eventually it was understood that if one
accepts Euclidean geometry, one must also accept the other.

Here is one of the models of non-Euclidean geometry used, often
called the Cayley model. Consider a disk in the Euclidean plane, and
forget all the points that are not inside this disk --- only the points
within the disk matter to us. Take two points, $A$ and $B$, connect
them with a straight line, and mark the points $X$ and $Y$, where this
line intersects the boundary circle.

Now, define the non-Euclidean distance between the points $A$ and $B$
by the following magical formula (which is not essential for what
follows%
\footnoteE{Note, however, that if one of the points travels toward the
boundary, the distance increases without bound. (Ed.)}):
\[
  \operatorname{distance}_{\textrm{noneucl}}(A,B)
  =
  \log\left(\frac{XB\cdot YA}{XA\cdot YB}\right)
\]
where $XB$, $YA$ etc., stand for the usual Euclidean distances between the
corresponding points.

\bigskip
\begin{lpic}[draft,l(10mm),r(10mm),b(10mm),clean]{Pix/cayley(0.2)}
  \lbl[rb]{0,90;$X$}
  \lbl[rb]{180,160;$Y$}
  \lbl[rb]{47,112;$A$}
  \lbl[rb]{107,138;$B$}
  \lbl[t]{100,-10; \parbox{0.4\textwidth}{\small Two points, $A$ and
      $B$, in the Cayley model.}}
\end{lpic}
\begin{lpic}[draft,clean,l(10mm),r(10mm),b(10mm)]{Pix/cayley-lines(0.2)}
  \lbl[rb]{57,25;$D$}
  \lbl[rb]{107,138;$P$}
  \lbl[t]{100,-10; \parbox{0.4\textwidth}{\small Many lines through
      point $P$ parallel to line $D$.}}
\end{lpic}


\bigskip

Well then, one can imagine some fictitious inhabitants
(mathematicians, perhaps?) living inside the disk, who would know
nothing of the world outside of it, nothing of Euclidean geometry. They
would teach their children that between any two points there exists a
certain distance --- which, of course, they would not call
``non-Euclidean'' since they would have no knowledge of Euclid.

These inhabitants would have a Ministry of Education, which would
write school curricula containing theorems in complete contradiction
with those of our Euclidean world!%
\footnoteA{The ministers of education in our world, on the other
  hand, tend to write curricula without any theorems at all\dots{} See the
  post by Valerio Vassallo at \url{
    https://images-archive.math.cnrs.fr/Requiem-pour-la-Geometrie.html}
  (in French)}
Moreover, it would seem obvious to them that through a point taken
outside a line, one can draw an infinite number of parallels.
Look at the figure above: the four lines passing through point $P$
appear parallel to line $D$, don’t they? And if you thought they
intersect $D$, I would reply that they meet outside the disk --- and
the outside of the disk simply does not exist in their world.
Thus, if we accept Euclidean geometry, we are compelled to accept
non-Euclidean geometry at the same time.

You can learn a little more about this new geometry in the
article ``Une chambre hyperbolique''.%
\footnoteE{Jos Leys, \textit{Une chambre hyperbolique},
  \url{https://images-archive.math.cnrs.fr/Une-chambre-hyperbolique.html}
(Ed.)}%
$^{,}$%
\footnoteE{See also \textit{Hyperbolic geometry: The first 150 years} by
  John Milnor, available at\\
  \url{https://projecteuclid.org/journals/%
    bulletin-of-the-american-mathematical-society-new-series/%
    volume-6/issue-1/Hyperbolic-geometry-The-first-150-years/%
    bams/1183548588.full}
  (Ed.)}

\section{Metric Spaces}

Starting from the beginning of the twentieth century, the concept of a
metric space appeared. We say that a set $E$ is a metric space (note
the \textit{indefinite} article, implying that there are many such
spaces --- obvious today, but a heresy not so long ago) if for each
pair of elements $x$, $y$ in $E$ (which we naturally call
\textit{points} of space $E$), there is a number called the distance
between $x$ and $y$.
This number must be non-negative (the least one could ask of a
distance), but it must also satisfy a few additional properties
(sometimes called axioms, though that word has lost some of its force
over time\dots{} Should it be ``axiom'' or ``definition''? --- not so clear
anymore).

\begin{itemize}[left=0pt]
\item $\d_{E}(x,y)=\d_{E}(y,x)$ for all points $x$, $y$.
  (Feel free to object: sometimes it’s harder to go from one point to
  another than the other way around --- for example, when climbing up a
  mountain path. But still, we’ll stick with this
  assumption/axiom/definition.)
\item $\d_{E}(x,y)=0$ if and only if $x=y$. (Ah yes, mathematicians
  can’t resist saying such incantations --- ``if and only if'', which
  they love to abbreviate as \textit{iff}) We agree: two points at
  zero distance are, in fact, one and the same point.
\item $\d_{E}(x,z)\leq\d_{E}(x,y)+\d_{E}(y,z)$ for any three points
  $x,y,z$ in $E$. This is \textit{the triangle inequality}.  It says
  that the path from $x$ to $z$ via $y$ is longer than the direct path
  from $x$ to $z$. Obvious? Perhaps. Yet most mathematicians are
  amazed at how Misha Gromov managed to make such a seemingly trivial
  axiom yield profound insights.
\end{itemize}

\section{How About Examples?}
Today, it’s almost obvious to most mathematics students that there are
many metric spaces. Barely two centuries ago only one was known ---
ours or at least the one Euclid had presented as ours.

So, here are a few examples nonetheless:
\begin{itemize}[left=0pt]
\item \textbf{The Non-Euclidean Plane.}  We’ve already encountered it:
  it’s the interior of the disk, with a distance defined by that
  mysterious formula that seems to have come out of nowhere\dots{}

\item \textbf{The Sphere.}  After all, we live on the surface of
  Earth. Just as we imagined a nation unaware of the world outside a
  disk, most of us (myself included) know nothing
  beyond what lies on Earth. The distance between two points on Earth
  is the length of the shortest path connecting them, while staying
  on Earth, of course.

  Anyone who has looked at the routes taken by airplanes crossing the
  Atlantic will understand that the geometry on Earth is not quite the
  same as Euclidean geometry.

\item \textbf{The Population of the Earth.}  Let’s think of all the
  men and women who are alive today, or who have lived in the
  past. That will be our set $E$. We define the distance between two
  persons as the length of the shortest path connecting them up in the
  family tree. The distance between a parent and their child is one
  unit.
  In general, the distance between two people is the length of the
  shortest chain linking them through parent/child connections ---
  father or mother $\to$ son or daughter.%
  \footnoteE{In this example we measure distances only through common
    ancestors, not common children or further descendants. (Ed.)}
  For example, the distance between my nephew and me is 3, since we
  must go up to my father, then down to my sister, and finally down to
  her son.

  This gives us a metric space whose geometry one might very much like
  to understand.

\item \textbf{The Internet.}  Here, the set $E$ consists of all the
  web pages on the planet. The distance between two pages is 1 if you
  can go from one to the other with a single click; it is equal to 2
  if you must pass through an intermediate page, and so on.  Note that
  in this case, the distance need not be symmetric: sometimes it’s
  easy to go from one page to another, but the reverse isn’t true.

  We could imagine many other communication networks as well ---
  Facebook, for instance, where the distance is the length of the
  shortest chain of friends connecting two people.
  
\item \textbf{French railway network.}  Here, the set $E$ is the
  French territory. One of its points is called Paris. We define the
  SNCF-distance%
  \footnoteE{SNCF (\textbf{S}oci\'et\'e \textbf{n}ationale des
    \textbf{c}hemins de \textbf{f}er fran\c{c}ais) is France' railway
    operator. French railway network is indeed almost star-shaped,
    centered at Paris, so that one is often forced to travel through
    Paris even if the straight line between the origin and destination
    doesn't even come close to the city. (Ed.)}
  between two points $A$ and $B$ as follows: If the straight line
  $(AB)$ does not pass through Paris, the SNCF-distance between $A$
  and $B$ is the sum of their usual (Euclidean) distances to Paris.
  On the other hand, if $(AB)$ does pass through Paris, the
  SNCF-distance is simply the usual Euclidean distance between $A$ and
  $B$.

  In short, this models a world where all routes go through Paris --- a
  playful geometric nod to the structure of the French railway
  network.
\end{itemize}
\medskip
There’s no need to continue the list --- it could be made endless. The
geometry of metric spaces became a central theme in mathematics during
the first half of the twentieth century.

\section{Hyperbolic Spaces}

Let us now turn to Misha Gromov, who was honored with the Abel Prize
in 2009 (see the special feature%
\footnoteE{Dossier : Mikha\"il Gromov, g\'eom\`etre,  \url{https://images-archive.math.cnrs.fr/+-Mikhail-Gromov-geometre-+.html}
  (in French) (Ed.)}
devoted to him on ``Images des Mathématiques''). Like many
mathematicians of his generation, he was fascinated by non-Euclidean
geometry. But what struck him most were the qualitative properties of
this remarkable geometry. He observed a kind of stability, which
I’ll try to explain below.

Gradually --- probably beginning in the early 1970s --- Gromov
observed a number of properties satisfied by non-Euclidean
geometry. He compared them, sought to understand which were essential
and which were incidental, and ultimately arrived at a fundamental
concept: that of \textit{the hyperbolic metric space} (often specified as \textit{in
the sense of Gromov}, since mathematicians have an unfortunate
tendency to label almost anything as hyperbolic).

The process was a long one. The text of his lecture%
\footnoteA{M. Gromov, \textit{Infinite groups as geometric objects},
  Proc. Int. Congress Math. Warsaw 1983 1 (1984), 385--392.}
at the International Congress in Warsaw (1983)%
\footnoteA{The Congress should have taken place in 1982, but was
  postponed for political reasons. Besides, Gromov was unable to attend
  it.}
already contained the major ideas, though still somewhat
unpolished. In 1987 his book on hyperbolic groups%
\footnoteA{M. Gromov, \textit{Hyperbolic groups}, in
\textit{Essays in Group Theory}, S. Gersten ed., MSRI Publications 8 (1987), 75--263
Springer.}
was published --- a work that, nearly a quarter of a century later,
can rightly be called marvelous. And yet, it was not easy to
read. Some criticized it for a certain lack of detail, but that can be
forgiven in a book intended as a \textit{qualitative} analysis of
geometry.

\section{An aside: The Style}

A brief digression is in order. Gromov’s style is unique, inimitable, and
far removed from mathematical convention. His writing is always
difficult, even irritating at times, often lacking in precision, yet
it always goes straight to the essence and possesses an incredible
density. His proofs are often only sketches; sometimes they are ``a little
wrong,'' but they always contain wonderful ideas.

Perhaps the best way to illustrate the originality of Gromov’s style
is simply to quote the opening lines of one of his articles:
\textit{Spaces and Questions}, GAFA, Geometric and Functional
Analysis, Special Volume (2000), 118--161.

\begin{quoting}[leftmargin=5mm,rightmargin=5mm]\sf
  Our Euclidean intuition, probably, inherited from ancient primates,
  might have grown out of the first seeds of space in the motor
  control systems of early animals who were brought up to sea and then
  to land by the Cambrian explosion half a billion years
  ago. Primates’ brain had been lingering for 30--40 million years.
  Suddenly, in a flash of one million years, it exploded into growth
  under relentless pressure of the sexual-social competition and
  sprouted a massive neocortex (70\% neurons in humans) with an
  inexplicable capability for language, sequential reasoning and
  generation of mathematical ideas. Then Man came and laid down the
  space on papyrus in a string of axioms, lemmas and theorems around
  300 B.C. in Alexandria.  Projected to words, brain’s space began to
  evolve by dropping, modifying and generalizing its axioms. First
  fell the Parallel Postulate: Gauss, Schweikart, Lobachevski, Bolyai
  (who else?) came to the conclusion that there is a unique
  non-trivial one-parameter deformation of the metric on
  $\mathbb{R}^{3}$ keeping the space fully homogeneous.
\end{quoting}

\section{A Definition}

It’s time to give some precise definitions. Gromov begins
from the following remarkable observation:

In a triangle in non-Euclidean geometry, the radius of the inscribed
circle cannot exceed a certain maximum value. Incredible --- because
this is, of course, completely false in Euclidean geometry: take a
large equilateral triangle, then its inscribed circle will also be
large. In non-Euclidean geometry, even if the sides of a triangle are
enormous, there exists a point (the center of the inscribed circle)
that remains close to each of the three sides.

Let’s start with a definition that isn’t actually essential, but will
allow us to speak about the sides of a triangle.

A metric space $E$ is called geodesic if, any two
points $x$  and $y$ can be connected by what we might naturally call
a \textit{segment} $xy$ --- that is, a curve $x(t)$ joining $x$ and $y$,
parametrized by a parameter $t$ varying between $0$ and the distance
$\d_{E}(x,y)$, such that the distance between $x(t)$ and $x(s)$
equals $|s-t|$.

This simply means that $x$ and $y$ are connected by a segment of the correct length. 
Thus, whenever we have three points $x,y,z$, we can
``draw'' the three sides $xy$, $yz$, and $zx$.\\
\begin{wrapfigure}{r}{0mm}
  \begin{lpic}[draft,t(5mm),l(5mm),r(5mm),b(-5mm),clean]{Pix/h-triangle(0.3)}
  \lbl[]{2.5,12;$\bullet$}
  \lbl[]{127,2;$\bullet$}
  \lbl[]{112,111;$\bullet$}
  \lbl[]{82,57;$\bullet$}
  \lbl[t]{-5,10;$x$}
  \lbl[lt]{130,0;$y$}
  \lbl[bl]{115,115;$z$}
  \lbl[tl]{73,61;$c$}
\end{lpic}
\end{wrapfigure}
\begin{definition}
Let $\delta\geq0$ be a real number (the Greek letter ``delta'' has
become the traditional symbol in this context).

A geodesic metric space $E$ is said to be $\delta$-hyperbolic if, for
every triangle formed by three points $x,y,z$, there exists a point $c$
that lies at a distance less than $\delta$ from each one of the three
sides of the triangle $xyz$. 

In other words, every triangle in $E$ is ``thin'' --- there is always a
point close to all three sides.
\end{definition}

\section{Two Examples and a (Very Small) Theorem of Gromov}

Of course, the whole game would be pointless if the non-Euclidean
plane of Bolyai–Lobachevsky–Gauss were not $\delta$-hyperbolic (for
some value of $\delta$). Indeed, it is possible to verify that the
non-Euclidean plane does possess the $\delta$-hyperbolicity property.

We now turn to another family of $\delta$-hyperbolic (even
$0$-hyperbolic) spaces, called \textit{trees}. The left figure below
shows a \textit{tree}. It contains a certain number of vertices, some
of which are connected by edges. The term \textit{tree} is reserved
for such a network provided it has no closed loops.


\begin{figure}
  \begin{lpic}[l(10mm),r(10mm),b(10mm),t(2mm)]{Pix/tree(0.3)}
    \lbl[t]{50,-10; A tree.}
  \end{lpic}
  \begin{lpic}[l(10mm),r(10mm),b(10mm),t(2mm)]{Pix/not-tree(0.3)}  
    \lbl[t]{50,-10; Not a tree.}
  \end{lpic}
\end{figure}

For example, the right figure above is not a tree.

For each edge of a tree, we can assign a non-negative length of our choosing. Then
the tree can be viewed as a geodesic metric space: the distance
between two points is simply the length of the shortest path
connecting them. Such a structure is called a \textit{metric tree}.

\begin{wrapfigure}{r}{0pt}
  \begin{lpic}[draft,t(5mm),l(5mm),r(5mm),b(5mm),clean]{Pix/t-triangle(0.3)}
    \lbl[t]{-5,10;$x$}
    \lbl[lt]{130,0;$y$}
    \lbl[bl]{115,115;$z$}
    \lbl[tl]{71,69;$c$}
  \end{lpic}
\end{wrapfigure}
What do triangles look like in a tree? The figure below shows that in
a tree, all triangles share a very distinctive property: there is
always a point common to all three sides! In a sense, one could say
that the inscribed circle has radius zero.
In any case, it should be clear that trees are 0-hyperbolic metric
spaces. Instead of a triangle, we more appropriately speak of a
tripod.


The following theorem is certainly not among the most difficult ones
proved by Gromov --- far from it! Its proof takes only a few lines in
one of his many books, and I’m sure he himself attaches only moderate
importance to it. 

Nevertheless, I present it here because it is quite typical of
Gromov’s qualitative approach to geometry and also because --- who
knows --- perhaps the readers might even be able to prove it on their
own?

There are many kinds of theorems. Some have statements that seem
fairly ordinary, but their difficulty lies in the proof, which
might not be easy. Others, by contrast, are quite easy to prove, and
the main challenge for their author was to find the right statement
--- the one that captures what truly matters.

\begin{theorem}
  Let $(E,\d_{E})$ be a $\delta$-hyperbolic geodesic metric space,
  and let $(x_{1},\dots,x_{n})$ be a collection of $n$ points in $E$. Then
  it is possible to construct a metric tree $A$, with its own distance
  function $\d_{A}$, and $n$ corresponding points $(y_{1},\dots,y_{n})$ in $A$,
  such that the distances in $A$ and in $E$ between the corresponding
  points are almost the same.

  More precisely, this means that for every pair $(i,j)$ the
  distances satisfy:
  \[
    |\d_{E}(x_{i},x_{j})-\d_{A}(y_{i},y_{j})| \leq 100\delta\cdot\log(n).
  \]

  In other words, any finite collection of points in a $\delta$-hyperbolic
  space can be approximated by points in a tree, up to a distortion
  bounded by $100\delta\cdot\log(n)$.
\end{theorem}

\section{A Few Remarks}

The above tree-approximation theorem is not particularly difficult, but it
provides a genuine intuition for hyperbolic geometry. If you want to
understand non-Euclidean geometry, look at trees, and you will at
least gain a qualitative understanding.

Why ``qualitative''? Because we study distances between points while
allowing for a certain uncertainty or error --- on the order
of $100\delta$. The goal is not to obtain precise measurements, but
rather to grasp the order of magnitude of things.

Gromov himself calls it \textit{coarse geometry}, that is, the geometry
of rough shapes.

Also note the $100\delta$ in the statement. The theorem is probably
still true if we replaced $100$ by $99$, but it’s not Gromov’s style
to chase after the best possible constants.

In his book, one finds here and there truly whimsical constants
(though rarely incorrect ones) of the kind $10\,000\delta$\dots{} In
some of his other articles, Gromov did not hesitate to use constants
whose numerical values are absolutely enormous.

\section{What’s the Point?}

All this would have only moderate interest if the following two
conditions were not met:
First of all, we have an excellent understanding of $\delta$-hyperbolic
spaces. Gromov’s book, several of his later papers, and other works
have revealed the full richness of these spaces. One can take as much
pleasure in ``doing geometry'' within these spaces as in our familiar
``old Euclidean space.'' There were proven many general theorems about
hyperbolic spaces.

Secondly, and above all, $\delta$-hyperbolic spaces abound, and that
is what makes them so interesting. We’ve already seen examples such as
trees (for instance, genealogical trees) and the non-Euclidean
plane. In a certain sense, one might even think that almost all spaces
are $\delta$-hyperbolic.

Examples arise from number theory, but also from group theory. As
early as 1990, Gromov even saw in this framework a possible
interpretation of certain growth processes in biology. A certain class
of groups introduced in the 1960s, known as \textit{small cancellation
  groups}, suddenly became much clearer thanks to this geometric
perspective --- geometry coming to the rescue of algebra, so to speak.

In this way, Misha Gromov identified a fundamental concept: that of
the hyperbolic group, whose meaning is above all geometric. But
of course, combining groups and geometry is hardly a new idea in
mathematics.

Much more recently, Gromov even introduced probabilistic arguments
into his geometric approach. He showed that if one picks a group at
random, there is a high probability that it will be hyperbolic.

On July 12, 1831, Gauss wrote a letter%
\footnoteE{Carl Friedrich Gauss, \textit{Werke, Band VIII},
  \url{https://link.springer.com/book/10.1007/978-3-642-92474-3}
  or
  \url{https://archive.org/details/briefwechselzwi03schugoog/page/n10/mode/2up}
  (Ed.)}
to his friend Schumacher about non-Euclidean geometry. Here is an English
translation of an excerpt from that letter.
\begin{quoting}[leftmargin=5mm,rightmargin=5mm]\sf
  \noindent
  \dots

  \noindent
  In non-Euclidean geometry, there are no similar figures without
  equality. For example, the angles of an equilateral triangle are not
  only dependent on $R$,%
  \footnoteE{By letter $R$, Gauss denotes the parameter of deformation
    of the geometry. Nowadays it is called (Gauss) curvature. (Ed.)}
  but also vary according to the size of the sides; and when
  the sides increases without limit, the angles can become as small as
  one wishes. It is therefore inherently contradictory to try to draw
  such a triangle by means of a scaled-down one --- one can, in
  essence, only designate it.

  \centerline{
    \begin{lpic}[b(12mm)]{Pix/gauss-triangles(0.5)}
      \lbl[t]{25,-3;\parbox{0.4\textwidth}{\raggedright
          Small non-Euclidean triangle.
          (\small\rm Captions added by the editor)}}          
      \lbl[t]{130,-3;\parbox{0.4\textwidth}{\raggedright
          The designation of a large non-Euclidean triangle.
          }}
    \end{lpic}
  }
  
  The designation of the infinite triangle in this sense would
  ultimately be

  \centerline{
    \begin{lpic}{Pix/gauss-t-triangle(0.5)}
    \end{lpic}
  }
  
  In Euclidean geometry there is nothing absolutely large, but in
  non-Euclidean geometry there is --- and this is precisely its
  essential character. Those who do not admit this thereby assume ipso
  facto the entirety of Euclidean geometry; but, as I have said, in my
  conviction this is mere self-deception.

  \noindent
  \dots
\end{quoting}

Incredible! Gauss explains that triangles in non-Euclidean geometry
look quite different from Euclidean triangles. He even says that, ``in
the limit'' they become triangles like those found in trees.

So, did Gauss already have a ``qualitative'' approach to geometry ---
a hundred and fifty years before Gromov? Of course, Gauss did not
develop a true qualitative theory of geometry as Gromov later did.

I don’t think Gromov ever saw this letter from Gauss, but I am
convinced that he would be proud to know that his path crossed the one
of Gauss.

\end{document}